%-------------------------%
%-----Document Setup------%
%-------------------------%
\documentclass[a4,10pt,oneside,openany]{memoir}
\usepackage[utf8x]{inputenc}
\usepackage[italian]{babel}
\usepackage{helvet}
\renewcommand{\familydefault}{\sfdefault}
\usepackage{savetrees}
\usepackage{verbatim}
\linespread{1.5}
\setlength{\footskip}{20pt}

%-------------------------%
%------Document Code------%
%-------------------------%
\newcommand{\thought}[1]{\textit{#1}}

\newcommand{\scenechange}{
  \par
  \vspace{\baselineskip}
  \par
\noindent}
%Creates a line break for a change of scene

\newcommand{\majorchange}{
  \par
  \vspace{\baselineskip}
  \hfill
  \textasteriskcentered
  \hfill
  \vspace{\baselineskip}
\noindent}
%creates a major line break, split by an asterisk for scene changes at the end of a page of where a sense of a major change is required.

%-------------------------%
%------Main Document------%
%-------------------------%
\begin{document}
\title{Appunti e racconti}
\author{Michele Spinella}
\date{\today}
\maketitle

\chapter*{Una giornata al Palazzo di Giustizia}
\begin{comment}
-------------------------
----------Plot-----------
-------------------------
Insert section plot from 1-plot.text here
-Section 1-
Max X words

----AIM----




--Details--




-------------------------
------Senses Check-------
-------------------------
Smell
Touch
Sound
Taste
Sight

-------------------------
------Other Checks-------
-------------------------
Checked adverb use? (0)
Checked cliche use? (0)
Checked tense integrity?
Checked perspective integrity?
Checked reuse of major words?
Checked sentence length?
Checked simile use? (<=5)
Checked metaphor use? (<=3)
Checked description length?
Checked paragraph density
\end{comment}

Bari è nel pieno del suo caos cronico, sono le undici e trentacinque, il sole è alto e scalda le vie cinte da alti caseggiati. Il traffico è caotico, tutti contro tutti, macchine, motorini e casse di verdura. La salvezza si è presentata come un garage già prenotato. Lo spartano ricovero di automobili era presidiato da un giovanotto con gli arti tatuati, barba curata nera portata con una certa arroganza, modi sbrigativi ma pragmatici e pantaloni corti. Dopo aver lasciato le chiavi nel cruscotto, come da indicazioni del giovane, scendo e faccio scendere i miei passeggeri, saluto ed esco dal portone.\newline
Uscendo dal portale, si viene gettati in un turbine di macchine e signore che trascinano carrelli della spesa gonfi di primizie regalate dal ricco entroterra. Si sente il rumore secco della parlata barese, a volte dolce a volte brusca e fastidiosa, tuttavia come chi la parla. Dopo essere passato tra fave fresche di stagione, un'edicola e una fitta nuvola di chiacchiere, appare il palazzo di giustizia. Enorme e potente si staglia su un quartiere che a detta di tutti brulica di delinquenti, che quindi non devono fare molta strada per andare a processo. Alto e possente come un corazziere ricorda a tutti i cittadini che la legge è dura ma è legge (\emph{Dura lex sed lex}).\newline
All'ingresso c'è un piantone che chiacchiera con un amico che gli sta proponendo qualche occasione, gli passo dinnanzi senza che lui si accorga di nulla, intento in questa discussione che sembra vitale. Davanti a noi ci sono quattro uomini che sembrano dei bravi manzoniani di aragonese memoria, ci scrutano e poi continuano nel loro chiacchiericcio. Sembrano non gradire l'atteggiamento di taluni avvocati che maneggiano i numeri con una calcolatrice. Inveiscono con espressioni dialettali beffarde, sarcastiche a tratti minacciose, la descrizione è semplice loro che schiacciano questi fastidiosi scribacchini con gli occhiali e la calcolatrice. Mi tornano in mente i giorni della Marina, quando dovevo continuamente guardarmi le spalle da loro simili, sempre con il taglierino in tasca e la mano veloce. Pensando a loro, mi accorgo che il cortile del tribunale è denso di macchine parcheggiate, in qualsiasi angolo ed in modalità che nemmeno uno stuntman esperto riuscirebbe a replicare. Tuttavia sono automobili di ogni ceto sociale e di ogni ordine e genere. Dall'utilitaria alla mercedes-benz, dal suv al piccolo tre ruote, quindi anche il parcheggio è un richiamo all'uguaglianza della giustizia, tutti i dipendenti hanno almeno un parcheggio.\newline
La facciata del gigante minaccioso è cinta da un'impalcatura, che vista dal parcheggio antistante sembra una garza sulla pancia di un paziente appena operato. Tuttavia la l'imponenza non ne risente affatto, come a dire ``\emph{...anche se ferito, guai ai criminali}''. I bravi intanto si dimenano in una discussione più animata, a tratti alzano la voce e si parlano sopra, a tratti ridono e a volte si incupiscono. I loro occhiali da sole tintinnano assieme all'abbondante materiale aureo che cinge i lori colli, i capelli lunghi scintillavano di brillantina al sole caldissimo, sembrano sempre bagnati di una sostanza oleosa. Vengo distolto dalla mia osservazione da ben più urgenti incombenze, dobbiamo entrare nel ventre pulsante, e ferito, della giustizia barese.\newline\scenechange
Passando sotto l'impalcatura e varcando la soglia dell'atrio ci si imbatteva su due trabiccoli di plastica, dietro queste due porte finte c'erano due giannizzeri travestiti da metronotte, che pretendevano lo svuotamento delle tasche da tutti gli effetti personali e addirittura di frugare dentro le borse. Dei due uno frugava e l'altro dirigeva il traffico di persone, dirottandole tutte verso suo maldestro collega che arrancava tra le mie chiavi e i pacchetti di sigarette di un "signore" che mi stava dietro e che spazientito voleva passare, tanto che quando varcai la soglia, mi scansò con un gesto della mano tanto fastidioso quanto arrogante.\newline
L'atrio subito mi apparve in tutta la sua maestà, pavimentato di marmo nero con venature bianche, due colonne enormi a sorreggere la monumentale struttura. Dei finestroni enormi davano luce a tutto il palazzo, nuvole di persone scorrevano da una parte all'altra e altre salivano le scale tutti assorti nei loro pensieri.

\chapter*{L'incontro imprevisto}
\begin{comment}
-------------------------
----------Plot-----------
-------------------------
Insert section plot from 1-plot.text here
-Section 1-
Max X words

----AIM----




--Details--




-------------------------
------Senses Check-------
-------------------------
Smell
Touch
Sound
Taste
Sight

-------------------------
------Other Checks-------
-------------------------
Verificare liceo Galilei a Milano città, verificare Gulf news ultimi numeri in cui si parla di daesh.
Verificare ospedale americano a Dubai
Verificare Microlite e verificare lukoil come si chiama ora e trovare un nome per la compagnia petrolifera russa. Avanti!
\end{comment}
\section*{Dubai}
\subsection*{L'imprevisto}
Quella mattina l'aria era irrespirabile. Da est soffiava un vento umido e caldissimo, un
misto di salsedine e sabbia. La colazione in albergo era sempre la stessa, frutta, dolci, toast; a lato c'erano anche delle pirofile contenenti dei cibi salati, per soddisfare le abitudini dei clienti americani o tedeschi. Alina mangi una brioche e bevve del caffè, dal sapore simile a quello solubile.
Raccolta una copia di "Gulf News" dalla mensola della lobby, si avviò verso l'uscita con il solito passo svelto, salutò il portiere in divisa ed uscì.
Subito sentì quel leggero mancamento, dovuto al passaggio dai venticinque gradi all'interno dell'albergo, ai quarantasei esterni, come una mano che prende la gola e stringe nè troppo piano nè troppo forte.
Passava in mezzo alle persone, pensando a quello che i russi quella mattina le avrebbero chiesto, pensava soprattutto alle lamentele che di certo volevano avanzare visto il ritardo nella consegna di alcuni moduli del software. Una situazione che poteva diventare critica visto il cliente avrebbe potuto chiedere il pagamento di una penale di diecimila euro per ogni giorno di ritardo, si ripromise di controllare attentamente lo stato dei lavori, una volta ritornata a Milano. Camminava tra quei palazzi semi-vuoti, per lo più alberghi di lusso, altissimi. Ad un tratto alzò gli occhi, si ritrovò di fronte una donna per un attimo le sembrò di essere di fronte ad uno specchio, ma i capelli dell'immagine riflessa erano nero corvino. Un brivido le corse lungo la schiena sentì le gambe cedere e poi il buio.\newline
All'ospedale americano il termostato posto nel corridoio segnava i canonici venticinque gradi Celsius. Alina aprì gli occhi e si trovò di fronte Maxim che nel suo inglese imparato all'Università  di Perm, gli chiese come si sentisse. Non riusciva a spiegare che cosa fosse accaduto, l'unica cosa che ricordava era quella donna esattamente uguale a lei, con la sola differenza dovuta al colore dei capelli. Il primo pensiero che le era venuto dopo che si era svegliata fu quello di aver avuto un'allucinazione dovuta all'eccessivo caldo, oppure uno sbalzo improvviso di pressione. Tuttavia una parte di lei, le stava dicendo che quello che aveva visto era reale, che quella donna era viva e vegeta ed era uguale a lei.\newline
Ma come era possibile lei era nata e cresciuta a Milano, suo padre era di origine ligure mentre la madre era di Padova, entrambi si trasferirono in Lombardia prima degli otto anni si conobbero tra i banchi del Liceo "Vittorio Veneto" di Milano, quando ancora era Regio liceo. Entrambi avevano viaggiato solo in Europa, al massimo nel 1987 a Cipro per una vacanza con le figlie, Alina e Chiara.\newline
Lei però di viaggi ne aveva fatti, da quando lavorava per la Microlite era stata in almeno 20 paesi del mondo, dal Canada alla Thailandia. La Microlite era un'azienda di consulenza informatica con sede legale nell'isola di Man, vantava trenta sedi nelle principali capitali mondiali, un fatturato che nessuna azienda di software avrebbe mai potuto raggiungere, per questi ed altri motivi in molti avrebbero voluto accedere alle strategie ed al codice sorgente che veniva prodotto.\newline
In quel momento tutti quei viaggi le stavano passando davanti agli occhi come se stesse cercando di rivedere tutti i luoghi, tutti gli alberghi e tutte le persone che in diciotto anni di carriera aveva visto, con cui aveva parlato e con cui aveva lavorato. Ma non riusciva a vedere nulla, non riusciva trovare nulla di strano o di fuori posto. Maxim la fissava inebetito, le disse che aveva parlato con il dottore e che l'indomani l'avrebbero dimessa. Il medico parlava di uno sbalzo pressorio dovuto al troppo caldo, una cosa comune tra gli stranieri che venivano a lavorare negli Emirati.\newline
Uscì dall'ospedale e chiamò un taxi, ancora frastornata per l'accaduto andò in albergo e dove si accasciò per cinque minuti sul letto, ripensando all'assurdità della situazione. La sede di Lukoil che fortunatamente era davanti l'albergo era un palazzo basso della "Dubai Properties", di colore chiaro e con finestre di vetro scuro che riflettevano la luce del sole. Maxim come al solito aveva quell'aria ansiosa con un accenno di disappunto, non riusciva a tradurre una e-mail che gli era arrivata da Londra, mentre Alexjiei, il tecnico di laboratorio, fissava come ipnotizzato l'andamento della temperatura sul profilo di un pozzo.\newline
Alina si ritrovò al centro dell'ufficio composto da quattro scrivanie, ciascuna in un angolo della stanza, Maxim alzò gli occhi e la salutò con la solita preoccupazione.
\scenechange
\subsection*{La riunione}
%Cosa fanno in questa riunione? UAT, evolutive cosa viene in mente ad Alina in questa riunione?
Il condizionatore della sala riunioni pompava nella stanza un'aria freddissima e secca, il termostato segnalava appena ventidue gradi Celsius. Gli unici che sembravano a loro agio erano i committenti russi, venuti in parte da Mosca ed in parte da Astana.

%add additional sections by creating new section#.tex files and copying the two lines above. You can number sections (remove the asterisks) or name them (in the curly braces) or even use chapters instead. Ooo, and don't forget you can do parts too!!
\chapter*{Il convitto}
Un raggio di sole entrava dalla finestra e scaldava la parete bianca, come ogni mattina il sole scaldava la stanza quanto bastava per permettere a Jessica e a Chiara di uscire da sotto le coperte senza raffreddarsi.
A turno andavano nel bagno della stanza, si cambiavano e poi subito a fare colazione con tutti gli altri. Entrando nel refettorio si sentiva subito l'odore del caffè latte, l'odore non era sgradevole forse faceva un po' impressione la marmitta enorme dal quale la signora lo pescava con il mestolo, per versarlo dentro le tazze tutte uguali.
Il refettorio era pavimentato di bianco,i tavoli di formica bianca e le panche di legno sembravano portate da una caserma. Tutto era minimale, tutto era necessario e non c'erano cose inutili, l'unica forma concessa era il rettangolo. Tutto era rettangolare, le panche i tavoli ed banco dal quale le signore distribuivano la colazione. Jessica li non ci voleva stare, già si vedeva a Milano, magari a studiare; aveva ottimi voti soprattutto in fisica. Gli insegnanti la ritenevano tra i migliori allievi del convitto, pur avendo frequentato le scuole medie a singhiozzo tra un affidamento e l'altro, riusciva sempre a risolvere tutti i problemi velocemente ed a volte con una certa creatività. Prediligeva le materie scientifiche a quelle letterarie, avrebbe potuto diventare un ottimo ricercatore. Ma quella mattina aveva deciso che doveva partire, a tutti i costi.
Dopo aver finto un malore riuscì a sgattaiolare dal pronto soccorso senza difficoltà, mentre l’educatrice chiacchierava con l’infermiera.\newline
Milano centrale era enorme e fredda, le pareti di marmo sembravano chiudersi sulle persone che scendevano le scale per schiacciarle come pomodori sotto un lastrone. Jessica scese veloce le scale e si avviò verso la metro, presto si accorse che non sarebbe riuscita ad eludere la sorveglianza ATM, indugiò e subito uscì. In superficie si trovò presto in viale della Repubblica, cominciò a camminare.

%%% Local Variables:
%%% mode: latex
%%% TeX-master: t
%%% End:

\end{document}

%%% Local Variables:
%%% mode: latex
%%% TeX-master: t
%%% End:
