\begin{document}
\section{La vita}
Walter Benjamin nasce a Berlino il 15 luglio 1892, da Emil, antiquario e mercante d'arte, e Paula Schönflies, di famiglia alto-borghese di origine ebraica. Dei suoi primi anni rimane il visionario scritto autobiografico degli anni Trenta Infanzia berlinese intorno al millenovecento. Dal 1905 per due anni si reca al "Landerziehungsheim" in Turingia, dove fa esperienza del nuovo modello educativo impartito da Gustav Wyneken, il teorico della Jugendbewegung, il movimento giovanile di cui Benjamin farà parte fino alla scoppio della Grande Guerra. Nel 1907 torna a Berlino, concludendo gli studi secondari nel 1912. In quello stesso anno comincia a scrivere per la rivista "Der Anfang", influenzata dalle idee di Wyneken. Dall'università di Berlino si trasferisce a quella di Friburgo in Bresgovia, dove, oltre a seguire le lezioni di Rickert, stringe un forte sodalizio col poeta Fritz Heinle, che morirà suicida due anni dopo. Scampato all'arruolamento dopo l'inizio della guerra, rompe con Wyneken, che aveva entusiasticamente aderito al conflitto. Nel 1915, trasferitosi a Monaco, dove segue i corsi del fenomenologo Moritz Geiger, conosce Gerschom Scholem, con cui inizia un'amicizia durata fino alla morte. L'anno dopo incontra Dora Kellner, che sposa nel 1917: dalla relazione nasce nel 1918 il figlio Stefan, quando la coppia si è ormai trasferita a Berna, dove Benjamin, già autore di importanti saggi ( Due poesie di Friedrich Hölderlin ; Sulla lingua in generale e sulla lingua degli uomini ), l'anno seguente si laurea in filosofia con Herbertz discutendo una tesi sul Concetto di critica d'arte nel Romanticismo tedesco . In Svizzera fa la conoscenza di Ernst Bloch, con cui avrà fino alla fine un rapporto controverso, tra entusiasmi e insofferenza. Nel 1920, tornato a Berlino, progetta senza successo la rivista Angelus Novus, scrive Per la critica della violenza e traduce Baudelaire. Nel 1923 conosce il giovane Theodor Adorno. Il suo matrimonio entra in crisi e nel 1924, durante un lungo soggiorno a Capri, conosce e s'innamora di Asja Lacis, una rivoluzionaria russa che lo induce ad avvicinarsi al marxismo. Pubblica un saggio su Le affinità elettive per la rivista di Hugo von Hoffmanstahl. Nel 1925 l'università di Francoforte respinge la sua domanda di abilitazione all'insegnamento accademico, accompagnata dallo scritto sull'Origine del dramma barocco tedesco, pubblicato infine tre anni dopo, insieme agli aforismi di Strada a senso unico. In questo periodo Benjamin si mantiene con la sua attività di critico e recensore per la "Literarische Welt" e traduttore (di Proust, con Franz Hessel) e viaggia tra Parigi e Mosca, cominciando a maturare il progetto (destinato a rimanere incompiuto) di un'opera sulla Parigi del XIX secolo (il cosiddetto Passagenwerk). Nel 1929 stringe un profondo rapporto con Brecht, che negli anni Trenta, dopo l'avvento del Terzo Reich, lo ospita a più riprese nella sua casa in Danimarca. Il 1933 segna infatti la definitiva separazione dalla Germania. Esule a Parigi, trascorre comunque lunghi periodi a Ibiza, Sanremo e Svendborg. Per la "Jüdische Rundschau" esce Franz Kafka, ma le sue condizioni economiche si fanno sempre più precarie: l'assegno garantitogli dallo "Zeitschrift für Sozialforschung" di Adorno e Horkheimer, per cui pubblica nel 1936 L'opera d'arte nell'epoca della sua riproducibilità tecnica e Eduard Fuchs, il collezionista e lo storico nel 1937, diventa il suo unico mezzo di sussistenza. Nel 1938-39 lavora su Baudelaire (Di alcuni motivi in Baudelaire), ma lo scoppio della seconda guerra mondiale lo induce a scrivere di getto il suo ultimo testo, le tesi Sul concetto di storia. Internato nel campo di prigionia di Nevers in quanto cittadino tedesco, viene rilasciato tre mesi dopo. Abbandona tardivamente Parigi e cerca di ottenere un visto per gli Stati Uniti. Nel settembre del 1940 viene bloccato alla frontiera spagnola dalla polizia: nella notte tra il 26 e il 27 si toglie la vita ingerendo una forte dose di morfina. Ai suoi compagni di viaggio fu concesso di passare il confine il giorno seguente.
\section{Il pensiero}
Benjamin è scrittore asistematico, privilegia la forma del saggio e dell'aforisma, e concepisce come compito specifico del critico il prendere posizione e la negazione dell'ordine esistente. Nei suoi lavori di critica letteraria riprende la pratica del commentario ebraico, diretta a restituire all'originale la forza distruttiva di cui neppure l'autore di esso era stato cosciente. Il linguaggio, infatti, ha funzione espressiva, non strumentale: attraverso di esso, l'uomo deve dare voce alle cose mute. Dunque, teoria critico-materialistica e pensiero utopico-messianico si congiungono in modo originale nell'opera di Benjamin. Nella genesi del suo pensiero sono presenti motivi della filosofia romantica (alla quale è dedicata la sua tesi di laurea sul Concetto di critica d'arte nel romanticismo tedesco , del 1918), il pensiero nietzscheano (per le critiche alle pretese sistematico-totalizzanti della ragione, l'atteggiamento ermeneutico critico nei confronti della tradizione culturale e della realtà sociale, l'attenzione per il rapporto tra i contenuti del pensare e i suoi modi espressivi), l'esperienza delle avanguardie artistico-letterarie (per tutto ciò che di che di rivoluzionario e di dirompente hanno avuto nei confronti di una concezione ottimistica-retorica dell'uomo). Una componente essenziale della formazione e del pensiero di Benjamin è poi il suo ebraismo, rivissuto in molti suoi aspetti (a cominciare dalla lacerante tensione tra attesa messianica e valorizzazione della memoria storica) attraverso il rapporto con Gershom Sholem, un grande studioso della mistica ebraica. E' al tema di una lingua pura, immediatamente simbolica (cui si oppone la violenza operata dall'astrazione e dal giudizio concettuale proprio delle moderne concezioni del pensiero e del linguaggio) che sono dedicati i primi saggi di Benjamin: Sulla lingua in generale e su quella degli uomini ( 1916 ); Per la critica alla violenza ( 1921 ); Il compito del traduttore ( 1923 ). Sull'interpretazione dell'opera d'arte è incentrato invece il Saggio sulle affinità selettive di Goethe ( 1924-1925 ). In esso s'annuncia un motivo decisivo della riflessione estetica di Benjamin: la conciliazione proposta o suggerita dall'opera d'arte è solo un'apparenza mistificante; quanto alla pretesa totalità essa è falsa e smentita dall'intima (benché talora non evidente) frammentarietà del prodotto artistico. Nell'opera d'arte non è immediatamente visibile una dimensione utopico-positiva. Questa semmai è presente nella forma dell'inespresso, "del non detto" dell'arte - ovvero in una speranza che peraltro possono solo cogliere solo coloro che ne sono radicalmente privi. L'opera più compiuta di Benjamin - la sola ch'egli potè portare a termine - è L'origine del dramma barocco tedesco ( 1928 ). Attraverso una ricca analisi delle forme e figure del dramma barocco (Trauerspiel) come impossibile tentativo di ripetere storicamente la tragedia greca, questo celebre saggio svolge un acuto e suggestivo discorso sui concetti di simbolo e allegoria - e più in generale sull'essere e sul conoscere umano. Benjamin presenta infatti l'allegoria barocca come critica dell'aspirazione classicista a riunificare la scissione originaria prodottasi nell'uomo ed espressa sia nella simbologia tecnologica (il creatore e la creatura, la caduta e la redenzione…), sia in alcune coppie antinomiche della tradizione occidentale (il finito e l'infinito, il sensibile e il sovrasensibile…). Sotto un diverso profilo, l'opera benjaminiana fornisce una chiave preziosa per interpretare anche alcune fondamentali aporie dell'arte (e della coscienza) moderna: Benjamin fa infatti vedere come la tensione a raggiungere nell'esperienza artistica il "simbolo" (e quindi l'unificazione effettiva di cosa, linguaggio e significato) esploda continuamente in "allegoria", ovvero in una dialettica eccentrica (priva di centro) tra quanto è figurato nell'espressione, le intenzioni soggettive che lo hanno prodotto e i suoi autonomi significati. Per questo scacco del simbolico la malinconia diviene, nell'indagine di Benjamin, il sentimento fondamentale del soggetto moderno. A un altro livello, ciò che il trionfo dell'allegoria rivela è un'insanabile lacerazione, una sempre più radicale perdita di senso, un decadimento dell'umano e della storia. A partire dagli anni '30 Benjamin si avvicinò in qualche misura alla "Scuola di Francoforte": pur senza mai entrare a far parte organica del gruppo, egli collaborò con la "Rivista per la ricerca sociale" ed ebbe un'intensa, seppur travagliata, amicizia con Adorno. Le molteplici differenze tra i due pensatori non debbono far dimenticare (come talora è accaduto) certe loro innegabili prossimità di interessi e anche, entro certi precisi limiti, di convinzioni teoriche. Sia Adorno sia Benjamin respingono il privilegiamento dell'esistente, la ubriV della ragione positivistica, la barbarie dell'organizzazione capitalistica e della società. Entrambi (ma soprattutto Benjamin) rifiutano un'interpretazione e una pratica della riflessione come ricerca del sistema, del fondamento assoluto. La filosofia, secondo loro, deve soprattutto mettere in luce le contraddizioni celate sotto le ingannevoli apparenze della realtà e, insieme, il bisogno di felicità e di emancipazione insito nel mondo umano. Tale bisogno si esprime (spesso in modo cifrato) nelle situazioni, nei testi, negli eventi più disparati. Per questo, entrambi i pensatori fanno filosofia interrogando le testimonianze o i segni più eterogenei e talvolta sconcertanti. Sotto tale profilo, il più caratteristico e suggestivo saggio di Benjamin è l'incompiuta opera su Parigi come " capitale del XIX secolo ", nella quale il pensatore ha cercato di afferrare il senso di un'intera epoca storica giustapponendo l'analisi della poesia di Baudelaire e quella dell'assetto urbanistico parigino, l'interpretazione di nuove figure psico-antropologiche (il "flaneur", il "dandy", la prostituta) e l'esame dei nuovi caratteri della produzione e della circolazione della merce. Molta attenzione egli dedica soprattutto alla figura di Baudelaire, di cui fu anche traduttore: in particolare, distingue il concetto di "esperienza" dal concetto di "esperienza vissuta"; la seconda permette di rielaborare razionalmente, attraverso la riflessione, gli "choc" della vita, così da impedirne la penetrazione nel profondo e da difenderne la coscienza dal loro assalto. La semplice "esperienza" è invece quella subita direttamente dallo choc, senza mediazione: è quest'ultimo il caso di Baudelaire, che nella vita cittadina subisce incessantemente l'esperienza degli choc prodotti dagli urti della folla, dalle luci, dalle novità dei prodotti e delle situazioni e insomma dall'esistenza stessa di una metropoli moderna. La folla sarebbe perciò la " figura segreta " (il suggello e insieme la potenza nascosta) della sua poesia: pur non essendo mai compiutamente rappresentata, tuttavia la folla è una presenza ossessiva nell'opera di Baudelaire e non va ricercata tanto nei temi e nei contenuti, quanto nella forma poetica, nel ritmo nervoso, ora ondulato, ora franto, del verso baudelairiano ( " questa folla, di cui Baudelaire non dimentica mai l'esistenza, non funse da modello a nessuna delle sue opere. Ma essa è iscritta nella sua creazione come figura segreta "). Nella propria anatomia della modernità, Benjamin si è spesso rivelato più aperto e spregiudicato di Adorno: ora interrogandosi sul fenomeno della droga, ora analizzando con simpatia produzioni socio-culturali in apparenza 'minori', come la letteratura per l'infanzia e il "feuilleton", la fotografia e i giocattoli. Un'altra e più sostanziale diversità fra i due filosofi è l'atteggiamento nei confronti dell'arte: convinto come Adorno che il fenomeno artistico sia un'esperienza particolarmente eloquente del disagio della civiltà, Benjamin ne ha una visione meno aristocratico-elitaria rispetto a quella dell'amico. Una significativa testimonianza di ciò è offerta dal saggio L'opera d'arte nell'epoca della sua riproducibilità tecnica (1936-37). In esso, Benjamin contrappone ad ogni interpretazione mistico-esoterica del fenomeno artistico una concezione in qualche modo secolarizzata di esso. Prodotto di uomini per altri uomini, l'arte va studiata " materialisticamente " sia nei suoi modi di elaborazione e di rappresentazione anche tecnica (non esclusi quelli fotografici e cinematografici) sai nelle particolari modalità percettive del suo fruitore. Lo sviluppo delle forze produttive, rendendo tecnicamente possibile la riproducibilità delle opere d'arte (pensiamo alla televisione, ai cd, alla radio, al computer, ecc), ha messo fine all'alone di unicità, originalità e irripetibilità dell'opera d'arte, ossia all' " aura " che la circonda di sacralità agli occhi della borghesia, la quale proietta in essa i suoi sogni e ideali aristocratici: l'aura è quindi l'alone ideale che rende sensibile al fruitore l'unicità irripetibile dell'atto creativo. Nella società di massa, in cui regna la riproducibilità dell'opera d'arte, l'opera d'arte " può introdurre la riproduzione dell'originale in situazioni che all'originale stesso non sono accessibili. In particolare, gli permette di andare incontro al fruitore, nella forma della fotografia o del disco. La cattedrale abbandona la sua ubicazione per essere accolta nello studio di un amatore d'arte; il coro che è stato eseguito in un auditorio oppure all'aria aperta può venir ascoltato in una camera. Ciò che vien meno è quanto può essere riassunto con la nozione di 'aura' e si può dire: ciò che vien meno nell'epoca della riproducibilità tecnica è l'aura dell'opera d'arte ". La riproducibilità tecnica segna il trionfo della copia e del " sempre uguale ", per uomini rimasti privi di saggezza; ma in ciò, secondo Benjamin, si annida un potenziale rivoluzionario, perché apre alle masse, soprattutto nelle forme del cinema e della fotografia, l'accesso all'arte e alle sue capacità di contestazione dell'ordine esistente. Solo attraverso la distruzione violenta di quest'ordine, ormai diventato inumano, si può aprire lo spazio per la redenzione e la felicità. Benjamin contesta le concezioni ottimistiche del progresso, condivise anche dal marxismo dei socialdemocratici tedeschi, secondo cui la storia è un cammino lineare di sviluppo crescente. Esse, infatti, si pongono dal punto di vista dei vincitori nella storia, anziché rimettere in questione le vittorie di volta in volta toccate alle classi dominanti. Si tratta, invece, di " spazzolare la storia contropelo ", strappandola al conformismo delle classi dominanti, ovvero accostandosi al passato come profezia di un futuro e arrestando la continuità storica con un salto e una rottura. Nella storia, infatti, non c'è un teloV , un "fine" garantito: e infatti anche sugli sviluppi della società sovietica Benjamin è pessimista. Solo recuperando e prendendo al proprio servizio la teologia e il messianesimo sarà possibile liberarsi dalla fede cieca in un progresso meccanico. La differenza più sostanziale tra Benjamin e Adorno è l'atteggiamento nei confronti del pensiero dialettico : profondo conoscitore ed estimatore della cultura tedesca, Benjamin 'ignora' Hegel. Il suo silenzio esprime un rifiuto che, lungi dal condannare i soli aspetti conciliativi/totalizzanti dell'hegelismo criticati anche da Adorno, investe la stessa concezione hegeliana dell'immanenza della ragione nel reale e, soprattutto, della storicità dialettico-progressiva di quest'ultimo. La critica benjaminiana dello storicismo (e, più in generale, della concezione moderna della temporalità e del suo senso) è radicale: la sua condanna Benjamin la esprime in "Tesi di filosofia della storia" (1940). Per Benjamin ogni rappresentazione del tempo/storia secondo moduli fisico/lineari è fuorviante: è falso, inoltre, che la storia sia un processo continuo e uniforme nel tempo; che tale processo sia accrescitivo e progressivo; che, quindi, i traguardi e le aspirazioni degli uomini si debbano necessariamente ed esclusivamente collocare 'davanti'. Alla redenzione umano/sociale si deve essere spinti, invece, dalla visione del passato, fatto di " rovine su rovine " e così orrendo da esercitare in chi (come l' Angelus Novus raffigurato in un acquerello di Paul Klee molto amato da Benjamin) sa voltarsi a guardarlo una spinta irresistibile verso un futuro diverso. Se il rifiuto di un tempo/storia monodimensionale e spaziale fa pensare a certe analoghe posizioni assunte da Bergson o da Dilthey, occorre subito aggiungere che Benjamin polemizza aspramente con tutti e due i filosofi. A suo avviso, la storia, ben lungi dall'essere riconducibile ad un' "Erlebnis" soggettiva, è qualcosa di estremamente oggettivo e corposo. Così oggettivo e corposo da costituire una realtà in larga misura estranea, o almeno 'altra' rispetto al soggetto. Sotto un certo aspetto, essa appare, come dicevamo, un " cumulo di macerie " , o anche un gioco di forze terribili, tanto più terribili in quanto sanno spesso mascherarsi sotto le forme di miti seducenti. Sotto un altro aspetto, essa contiene però princìpi e valori non solo preziosi, ma imprescindibili e insostituibili. Purtroppo, non sempre il presente vuole e sa interrogare il tempo che è stato: soltanto certe epoche riescono ad inoltrarsi per tale itinerario interrogativo; e solo in certi casi si riesce ad entrare in rapporto con ciò cui, più o meno consapevolmente, si tende. Ma la ricerca di questo rapporto è un compito al quale non ci si può e non ci si deve sottrarre: la decifrazione del passato consente infatti di cogliere e di rivitalizzare idee e "unità di senso" che erano rimaste come se sepolte e bloccate nei loro possibili sviluppi. Inoltre, le domande che rivolgiamo al passato sono in realtà le nostre domande: solo comprendendo il passato comprendiamo noi stessi. Solo liberandone le virtù nascoste liberiamo noi stessi. Il Novecento appare a Benjamin abitata da grandi potenzialità sia positive (le possenti spinte auto-emancipatorie degli oppressi) sia negative (i totalitarismi, il potere tecnologico non adeguatamente controllato). In veste di marxista sui generis , Benjamin sostiene la necessità che le classi rivoluzionarie sappiano svolgere approssimativamente il loro compito teorico e pratico: senza cullarsi nell'illusione di riforme graduali e indolori, senza sottomettersi ai miti del progresso e della tecnica, ma assumendo invece una responsabilità 'epocale': quella di capire e di far capire che viviamo in uno " stato di emergenza ". Nelle Tesi di filosofia della storia , composte negli ultimi mesi della sua vita in Francia, Benjamin si richiama (a partire dal titolo) alle 11 Tesi su Feuerbach di Marx: in esse, Benjamin conduce una dura critica nei confronti dello storicismo, che giustifica gli eventi storici e assume quindi il punto di vista di coloro che hanno vinto nella storia. Egli indica, invece, una possibilità di vittoria per il materialismo storico, se questo " prende al suo servizio la teologia ", che oggi " è piccola e brutta ". Il recupero della tradizione messianica consente infatti di concepire il tempo come un processo non lineare, bensì solcato da improvvisi istanti rivoluzionari che frantumano la continuità storica: " la coscienza di far saltare il 'continuum' della storia è propria delle classi rivoluzionarie nell'attimo della loro azione. […] Al concetto di un presente che non è passaggio, ma in bilico nel tempo ed immobile, il materialista storico non può rinunciare. Poiché questo concetto definisce appunto il presente in cui egli per suo conto scrive la storia. Lo storicismo postula un'immagine eterna del passato, il materialista storico un'esperienza unica con esso. Egli lascia che altri sprechino le proprie energie con la meretrice 'C'era una volta' nel bordello dello storicismo. Egli rimane signore delle sue forze: uomo abbastanza per far saltare il 'continuum' della storia ". 
\end{document}