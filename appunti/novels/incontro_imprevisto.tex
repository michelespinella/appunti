\begin{comment}
-------------------------
----------Plot-----------
-------------------------
Insert section plot from 1-plot.text here
-Section 1-
Max X words

----AIM----




--Details--




-------------------------
------Senses Check-------
-------------------------
Smell
Touch
Sound
Taste
Sight

-------------------------
------Other Checks-------
-------------------------
Verificare liceo Galilei a Milano città, verificare Gulf news ultimi numeri in cui si parla di daesh.
Verificare ospedale americano a Dubai
Verificare Microlite e verificare lukoil come si chiama ora e trovare un nome per la compagnia petrolifera russa. Avanti!
\end{comment}
\section*{Dubai}
\subsection*{L'imprevisto}
Quella mattina l'aria era irrespirabile. Da est soffiava un vento umido e caldissimo, un
misto di salsedine e sabbia. La colazione in albergo era sempre la stessa, frutta, dolci, toast; a lato c'erano anche delle pirofile contenenti dei cibi salati, per soddisfare le abitudini dei clienti americani o tedeschi. Alina mangi una brioche e bevve del caffè, dal sapore simile a quello solubile.
Raccolta una copia di "Gulf News" dalla mensola della lobby, si avviò verso l'uscita con il solito passo svelto, salutò il portiere in divisa ed uscì.
Subito sentì quel leggero mancamento, dovuto al passaggio dai venticinque gradi all'interno dell'albergo, ai quarantasei esterni, come una mano che prende la gola e stringe nè troppo piano nè troppo forte.
Passava in mezzo alle persone, pensando a quello che i russi quella mattina le avrebbero chiesto, pensava soprattutto alle lamentele che di certo volevano avanzare visto il ritardo nella consegna di alcuni moduli del software. Una situazione che poteva diventare critica visto il cliente avrebbe potuto chiedere il pagamento di una penale di diecimila euro per ogni giorno di ritardo, si ripromise di controllare attentamente lo stato dei lavori, una volta ritornata a Milano. Camminava tra quei palazzi semi-vuoti, per lo più alberghi di lusso, altissimi. Ad un tratto alzò gli occhi, si ritrovò di fronte una donna per un attimo le sembrò di essere di fronte ad uno specchio, ma i capelli dell'immagine riflessa erano nero corvino. Un brivido le corse lungo la schiena sentì le gambe cedere e poi il buio.\newline
All'ospedale americano il termostato posto nel corridoio segnava i canonici venticinque gradi Celsius. Alina aprì gli occhi e si trovò di fronte Maxim che nel suo inglese imparato all'Università  di Perm, gli chiese come si sentisse. Non riusciva a spiegare che cosa fosse accaduto, l'unica cosa che ricordava era quella donna esattamente uguale a lei, con la sola differenza dovuta al colore dei capelli. Il primo pensiero che le era venuto dopo che si era svegliata fu quello di aver avuto un'allucinazione dovuta all'eccessivo caldo, oppure uno sbalzo improvviso di pressione. Tuttavia una parte di lei, le stava dicendo che quello che aveva visto era reale, che quella donna era viva e vegeta ed era uguale a lei.\newline
Ma come era possibile lei era nata e cresciuta a Milano, suo padre era di origine ligure mentre la madre era di Padova, entrambi si trasferirono in Lombardia prima degli otto anni si conobbero tra i banchi del Liceo "Vittorio Veneto" di Milano, quando ancora era Regio liceo. Entrambi avevano viaggiato solo in Europa, al massimo nel 1987 a Cipro per una vacanza con le figlie, Alina e Chiara.\newline
Lei però di viaggi ne aveva fatti, da quando lavorava per la Microlite era stata in almeno 20 paesi del mondo, dal Canada alla Thailandia. La Microlite era un'azienda di consulenza informatica con sede legale nell'isola di Man, vantava trenta sedi nelle principali capitali mondiali, un fatturato che nessuna azienda di software avrebbe mai potuto raggiungere, per questi ed altri motivi in molti avrebbero voluto accedere alle strategie ed al codice sorgente che veniva prodotto.\newline
In quel momento tutti quei viaggi le stavano passando davanti agli occhi come se stesse cercando di rivedere tutti i luoghi, tutti gli alberghi e tutte le persone che in diciotto anni di carriera aveva visto, con cui aveva parlato e con cui aveva lavorato. Ma non riusciva a vedere nulla, non riusciva trovare nulla di strano o di fuori posto. Maxim la fissava inebetito, le disse che aveva parlato con il dottore e che l'indomani l'avrebbero dimessa. Il medico parlava di uno sbalzo pressorio dovuto al troppo caldo, una cosa comune tra gli stranieri che venivano a lavorare negli Emirati.\newline
Uscì dall'ospedale e chiamò un taxi, ancora frastornata per l'accaduto andò in albergo e dove si accasciò per cinque minuti sul letto, ripensando all'assurdità della situazione. La sede di Lukoil che fortunatamente era davanti l'albergo era un palazzo basso della "Dubai Properties", di colore chiaro e con finestre di vetro scuro che riflettevano la luce del sole. Maxim come al solito aveva quell'aria ansiosa con un accenno di disappunto, non riusciva a tradurre una e-mail che gli era arrivata da Londra, mentre Alexjiei, il tecnico di laboratorio, fissava come ipnotizzato l'andamento della temperatura sul profilo di un pozzo.\newline
Alina si ritrovò al centro dell'ufficio composto da quattro scrivanie, ciascuna in un angolo della stanza, Maxim alzò gli occhi e la salutò con la solita preoccupazione.
\scenechange
\subsection*{La riunione}
%Cosa fanno in questa riunione? UAT, evolutive cosa viene in mente ad Alina in questa riunione?
Il condizionatore della sala riunioni pompava nella stanza un'aria freddissima e secca, il termostato segnalava appena ventidue gradi Celsius. Gli unici che sembravano a loro agio erano i committenti russi, venuti in parte da Mosca ed in parte da Astana.
