\documentclass[12pt, a4paper]{article}
\usepackage[utf8]{inputenc}
\usepackage[a4paper,margin=0.75in]{geometry}
\usepackage[T1]{fontenc}
\usepackage{setspace}


\author{Michele Spinella \thanks{Dolo,13/11/1976}}
\date{Dicembre 2019}


\title{Appunti per racconti}

\maketitle

\begin{document}

\ttfamily
\onehalfspacing


\section*{La scuola}
% TODO Da dove viene il bambino? 
La scuola era alla fine di una strada in salita, asfaltata da poco. Anni addietro infatti era usata per salire alla malga. Attorno alla scuola c'erano dei prati verdi, alcuni in salita  altri con delle piccole piane, dove spesso i bambini giocavano.
Era di carnagione più scura, i capelli crespi e gli occhi neri. Saliva con fatica la strada, e si guardava attorno, come se si fosse perso. Gli altri bambini lo superavano sicuri, accompagnati dai genitori, che li spronavano  come se stessero portando le capre al pascolo.
Era la prima volta che faceva quella strada e già non gli piaceva, era ripida.
Da dove veniva non c'erano salite così ripide, c'erano il mare, gli scogli e il vento. Un vento che portava l'odore dei pini marittimi e li faceva sussurrare, portava il canto delle cicale quando faceva caldo.
Niente di tutto ciò. L'aria era fredda e pioveva ogni sera, la casa era molto umida. La mamma il pomeriggio faticava ad accendere il fuoco, anche la legna era impregnata di umidità.
\newline
Sulla scia di questi pensieri era arrivato a scuola, si ritrovò sulla soglia, sembrava che le gambe lo avessero trasportato da sole fino a li. Alzò la testa e si ritrovò di fronte ad un uomo alto e calvo, con un grembiule blu scuro. L'uomo gli chiese in che classe dovesse andare, gli rispose chiaramente "in seconda C", alzando la testa.
Guardandosi attorno notò che tutti gli altri bambini erano accompagnati dalle mamme o dai papà, solo lui era da solo ed accompagnato da quell'uomo indifferente ed ostile.\newline
Gli altri bambini erano già  in classe, seduti, con i quaderni e gli astucci nuovissimi disposti sul banco. Lui invece aveva un quaderno a quadretti, qualche penna e alcuni pastelli. Aprì il quaderno come faceva sempre la sera a casa, mentre la mamma curva nel tinello lavava i piatti o puliva della verdura. Il quaderno si aprì dove era messo come segno una matita,  con una mano appiattì il foglio e continuò a colorare quel tacchino che aveva disegnato il giorno prima.

\end{document}
%%% Local Variables:
%%% mode: latex
%%% TeX-master: t
%%% End:
