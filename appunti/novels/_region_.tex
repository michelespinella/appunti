\message{ !name(lascuola.tex)}\documentclass[novel]{sffms}


\author{Michele Spinella}
\title{La Scuola}

\runningtitle{La Scuola}
\surname{Spinella}


\begin{document}

\message{ !name(lascuola.tex) !offset(-3) }


\begin{synopsis}
Un bambino va a scuola ma non � accettato da nessuno perch� forestiero.
\end{synopsis}

\chapter{Uno}
% TODO Da dove viene il bambino? 
La scuola era alla fine di una strada in salita, asfaltata da poco. Anni addietro infatti era usata per salire alla malga. Attorno alla scuola c'erano dei prati verdi, alcuni in salita  altri con delle piccole piane, dove spesso i bambini giocavano.
Era di carnagione pi� scura, i capelli crespi e gli occhi neri. Saliva con fatica la strada, e si guardava attorno, come se si fosse perso. Gli altri bambini lo superavano sicuri, accompagnati dai genitori, che li spronavano  come se stessero portando le capre al pascolo.
Era la prima volta che faceva quella strada e gi� non gli piaceva, era ripida.
Da dove veniva non c'erano salite cos� ripide, c'erano il mare, gli scogli e il vento. Un vento che portava l'odore dei pini marittimi e li faceva sussurrare, portava il canto delle cicale quando faceva caldo.
Niente di tutto ci�. L'aria era fredda e pioveva ogni sera, la casa era molto umida. La mamma il pomeriggio faticava ad accendere il fuoco, anche la legna era impregnata di umidit�.
\newline
Sulla scia di questi pensieri era arrivato a scuola, si ritrov� sulla soglia, sembrava che le gambe lo avessero trasportato da sole fino a li. Alz� la testa e si ritrov� di fronte ad un uomo alto e calvo, con un grembiule blu scuro. L'uomo gli chiese in che classe dovesse andare, gli rispose chiaramente "in seconda C", alzando la testa.
Guardandosi attorno not� che tutti gli altri bambini erano accompagnati dalle mamme o dai pap�, solo lui era da solo ed accompagnato da quell'uomo indifferente ed ostile.

\end{document}
%%% Local Variables:
%%% mode: latex
%%% TeX-master: t
%%% End:

\message{ !name(lascuola.tex) !offset(-37) }
