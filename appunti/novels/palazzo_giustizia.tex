\begin{comment}
-------------------------
----------Plot-----------
-------------------------
Insert section plot from 1-plot.text here
-Section 1-
Max X words

----AIM----




--Details--




-------------------------
------Senses Check-------
-------------------------
Smell
Touch
Sound
Taste
Sight

-------------------------
------Other Checks-------
-------------------------
Checked adverb use? (0)
Checked cliche use? (0)
Checked tense integrity?
Checked perspective integrity?
Checked reuse of major words?
Checked sentence length?
Checked simile use? (<=5)
Checked metaphor use? (<=3)
Checked description length?
Checked paragraph density
\end{comment}

Bari è nel pieno del suo caos cronico, sono le undici e trentacinque, il sole è alto e scalda le vie cinte da alti caseggiati. Il traffico è caotico, tutti contro tutti, macchine, motorini e casse di verdura. La salvezza si è presentata come un garage già prenotato. Lo spartano ricovero di automobili era presidiato da un giovanotto con gli arti tatuati, barba curata nera portata con una certa arroganza, modi sbrigativi ma pragmatici e pantaloni corti. Dopo aver lasciato le chiavi nel cruscotto, come da indicazioni del giovane, scendo e faccio scendere i miei passeggeri, saluto ed esco dal portone.\newline
Uscendo dal portale, si viene gettati in un turbine di macchine e signore che trascinano carrelli della spesa gonfi di primizie regalate dal ricco entroterra. Si sente il rumore secco della parlata barese, a volte dolce a volte brusca e fastidiosa, tuttavia come chi la parla. Dopo essere passato tra fave fresche di stagione, un'edicola e una fitta nuvola di chiacchiere, appare il palazzo di giustizia. Enorme e potente si staglia su un quartiere che a detta di tutti brulica di delinquenti, che quindi non devono fare molta strada per andare a processo. Alto e possente come un corazziere ricorda a tutti i cittadini che la legge è dura ma è legge (\emph{Dura lex sed lex}).\newline
All'ingresso c'è un piantone che chiacchiera con un amico che gli sta proponendo qualche occasione, gli passo dinnanzi senza che lui si accorga di nulla, intento in questa discussione che sembra vitale. Davanti a noi ci sono quattro uomini che sembrano dei bravi manzoniani di aragonese memoria, ci scrutano e poi continuano nel loro chiacchiericcio. Sembrano non gradire l'atteggiamento di taluni avvocati che maneggiano i numeri con una calcolatrice. Inveiscono con espressioni dialettali beffarde, sarcastiche a tratti minacciose, la descrizione è semplice loro che schiacciano questi fastidiosi scribacchini con gli occhiali e la calcolatrice. Mi tornano in mente i giorni della Marina, quando dovevo continuamente guardarmi le spalle da loro simili, sempre con il taglierino in tasca e la mano veloce. Pensando a loro, mi accorgo che il cortile del tribunale è denso di macchine parcheggiate, in qualsiasi angolo ed in modalità che nemmeno uno stuntman esperto riuscirebbe a replicare. Tuttavia sono automobili di ogni ceto sociale e di ogni ordine e genere. Dall'utilitaria alla mercedes-benz, dal suv al piccolo tre ruote, quindi anche il parcheggio è un richiamo all'uguaglianza della giustizia, tutti i dipendenti hanno almeno un parcheggio.\newline
La facciata del gigante minaccioso è cinta da un'impalcatura, che vista dal parcheggio antistante sembra una garza sulla pancia di un paziente appena operato. Tuttavia la l'imponenza non ne risente affatto, come a dire ``\emph{...anche se ferito, guai ai criminali}''. I bravi intanto si dimenano in una discussione più animata, a tratti alzano la voce e si parlano sopra, a tratti ridono e a volte si incupiscono. I loro occhiali da sole tintinnano assieme all'abbondante materiale aureo che cinge i lori colli, i capelli lunghi scintillavano di brillantina al sole caldissimo, sembrano sempre bagnati di una sostanza oleosa. Vengo distolto dalla mia osservazione da ben più urgenti incombenze, dobbiamo entrare nel ventre pulsante, e ferito, della giustizia barese.\newline\scenechange
Passando sotto l'impalcatura e varcando la soglia dell'atrio ci si imbatteva su due trabiccoli di plastica, dietro queste due porte finte c'erano due giannizzeri travestiti da metronotte, che pretendevano lo svuotamento delle tasche da tutti gli effetti personali e addirittura di frugare dentro le borse. Dei due uno frugava e l'altro dirigeva il traffico di persone, dirottandole tutte verso suo maldestro collega che arrancava tra le mie chiavi e i pacchetti di sigarette di un "signore" che mi stava dietro e che spazientito voleva passare, tanto che quando varcai la soglia, mi scansò con un gesto della mano tanto fastidioso quanto arrogante.\newline
L'atrio subito mi apparve in tutta la sua maestà, pavimentato di marmo nero con venature bianche, due colonne enormi a sorreggere la monumentale struttura. Dei finestroni enormi davano luce a tutto il palazzo, nuvole di persone scorrevano da una parte all'altra e altre salivano le scale tutti assorti nei loro pensieri.
