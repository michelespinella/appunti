\section*{Una mattinata ai macelli (Carlo Emilio Gadda}
I segni si rincorrono lungo la pista dello Zodiaco: già lo Scorpione abbranca il piatto della fuggitiva Bilancia. La città, vorace acquirente, alletta al suo mercato indefettibile commissionari e negozianti di porci, mediatori, macellari ed augusti bovari. È la più popolosa del nord, una delle più ricche, attivissima. Chi non mangia, non lavora. Qualcosa, in pentola, deve bollire ad ogni costo: perché il martello abbia a cader pieno sul ferro o adempiersi a un cenno lo smistamento dei veicoli indemoniati, senza urti, senza risucchi.

La città si sveglia. Contro il sole già alto le case si levano bianche, ognuna per suo conto, (1) quasi ammodernate torri, dal verde vivido della pianura, che appare sottilmente ovattata dalle prime sue nebbie: i treni rallentano la lunga corsa sopra i canali e le rogge, lungo gli stendimenti di infaticabili lavandai.

Le linee elettriche ad altissima tensione sorpassano i pioppi, accostano l’agglomerato delle case e delle fabbriche fino alle sottostazioni periferiche: ivi si disarmano, (2) come l’armato potere dei consoli davanti la silente legge e le porte dell’Urbe. Gli apparecchi di Taliedo già ronzano, con le ali ombrate o dorate, sopra la testa degli spazzini insonnoliti; rientrano pedalando lenti i guardiani della notte, con una sigaretta tra le labbra; i gatti salutano il giorno accoccolandosi presso la macchina dell’espresso, nelle più mattutine tabaccherie. Un andirivieni di biciclette senza incrocio possibile.

La città chiede bovi, porci e vitelli a chi li ha saputi allevare. Grossi autocarri li sbarcano dalla verde provincia, da Cremona, da Mantova, da Stradella, dal Lodigiano, dall’Emilia e dal Veneto: qualche carretta lunga, con uno o due capi, arriva di qui presso. Partiti avanti l’alba con dodici capi, e dodici dentro il rimorchio, ecco già si spalancano sulla banchina; e ne fuorescono sull’ammattonato i fessìpedi a ritrovare la luce, la sicurezza ferma del suolo. Incedono verso il veterinario bianco nella dignità della loro natura e delle lor forme, odorosi di vita: dopo la breve sosta alle barre, i «cacitt» li sospingono fuori del recinto di sbarco, (usando bastoncelli di frassino, come corte fruste impugnate alla rovescia), avviandoli verso la pesatura e le stalle.

Vedo la strapazzata masnada attendere nei posti di arrivo l’esame del veterinario, uno dopo l’altro, poi decèdere con qualche blando muggito lungo il piano inclinato della banchina: uno relutta, o s’adombra, si rivolge sui passi già fatti, costringe impaurito all’inseguimento, per tutto il piazzale, gli uomini dal bastone e dalla tunica blu, che lo rincorrono e lo prevengono, con vociferazioni e agitazioni delle braccia.

La nuova paura vince l’altra, e ripiglia il cammino prescritto. Nell’attesa del medico qualche animale appoggia la fronte a una barra (bavando una sua schiuma dalla bocca, a fiocchi) quasi per raggelare al contatto del ferro, dopo la scombussolata notte, il tumulto doloroso del proprio sangue.

Qualche altro ha un corno mezzo divelto, e ne sanguina: il caglio scarlatto gli si è raggrumato giù per il muso, l’occhio immalinconito sembra dimandarne la cagione alle cose, al mondo. I «caccini» dalla tunica blu sono uomini tozzi, tra lo stalliere e il bovaro: hanno una placca d’ottone sul petto, col numero, come i facchini delle stazioni.

Loro cómpito è guidare e sorvegliare i bovi dalla banchina alle stalle di sosta, alle pese, al dazio, al macellatoio, lungo l’intera percorrenza: ogni bestia paga un tanto, a forfait.

Il veterinario della Sanità Municipale eseguisce, come detto, una prima ispezione allo sbarco. Certi stallieri vuotano gli autocarri ed i traini dalla paglia trita e dallo strame notturno, l’ammucchiano in appositi padiglioni. Altri, nello spiazzo di ricevimento, lavano carri e autocarri con getti d’acqua.

Intanto anche un treno è arrivato: poiché la città compera dovunque il suo lesso, cliente ottima dei pascoli e di lontane foraggiature: da Postumia entrano i bovini di Croazia e d’Ungària. Tanti ne entrano, che il mercato degli animali in pianta s’è quasi trasferito colà.

Da un prossimo scalo ferroviario, che serve e disserve tutta l’annona milanese, la locomotiva della Direzione Macello (pare una vecchietta gobba, ma basta al suo compito) ha trainato il convoglio lamentoso fino alla banchina: i quadrupedi ne escono mezzo intontiti, digiuni: alcuni paiono infreddoliti, rattrappiti: con deboli gambe sotto il gravame della testa, delle anche e delle culatte. Il loro incedere è più peso del solito, timido e malsicuro.

Vedo che non tutti i cornuti hanno ricevuto quelle cure privative cui si sommettono i vitellini, per farne dei manzi che siano veramente degni di Milano. Per i piani inclinati discendono dalla banchina, lunghissima, tori insigni, i quali procedono a fatica pure all’ingiù, con la gravità decorosa di chi si sente onusto d’evidenti benemerenze. Le gambe di dietro paiono aver perduto l’articolazione del ginocchio: e sono esse la vera e l’unica causa del ritardo.

Ciò mi illumina circa il gran lavorare che ho fatto — tante volte! — a tavola. Masticavo, masticavo, con la soddisfazione di una molazza, in cartiera, che digerisca la resa d’un romanzo-toro.

Ecco le pesatrici automatiche: allineate in batteria sotto una pensilina in calcestruzzo armato, a chiusura del piazzale: ognuna la sua chiara cabina: ognuna è provveduta di un’aletta d’entrata senza ritorno, un po’ come i conta-persone dei musei; ma ci passa un bel bove.

Tutti gli sbarramenti d’avvìo e di raccolta sono in tubo di ferro verniciato di grigio: compiutisi il ricevimento e la conta, subito il personale di pulizia subentra a quell’altro, con ramazze e manichette ad acqua: per detergere la banchina e il piazzale.

Interrogata, ogni pesatrice enuncia il peso dell’animale su talloncini a stampa, e il responso determina il costo. I commissionari, (in rappresentanza del negoziante), e i macellai acquirenti presenziano la breve cerimonia.

Talora i bovini arrivano con qualche anticipo, da venti a sessanta ore, ed è ovvio, rispetto al giorno di macellazione: in tal caso vengono stabulati in ampie e chiare stalle, pagando un forfait per giorno e per capo. Ma per lo più dopo la prima pesatura, vengono avviati a quella fiscale del Dazio, indi ai padiglioni di macello.

Ne seguo il muto brancolamento, contenendo l’angoscia, il malessere. Mi dico e mi ripeto che si tratta di una necessità senza alternativa, il luogo, nel sole tepido, non è altra cosa se non un mercato, uno «stabilimento» qualunque…

I vitelli vengono trasportati alla loro fine su carri speciali, trainati da carrelli ad accumulatori. Tristi e direi présaghi, paralizzati in una rassegnazione senza più gemiti, ne vengono fatti discendere a quattro a quattro per una specie di barcarizzo e vi slittano come semplici pesi, qualcuno a culo indietro, piovendo entro i brevi recinti di entrata dell’ammazzatoio.

Qualche cosa di simile, più in là, deve accadere ai porcelli, clamorosi e striduli, inutilmente striduli.

Sospinti dai caccini, i buoi ed i tori arrivano invece con le loro gambe, lentamente, alla fortuna scarlatta. Entrano nel padiglione pavimentato di piastrelle rosse, diretti dalle stangate sempre più tenui e quasi oramai fatte pietose degli uomini dalla tunica blu: un uomo li attende, con una tunica blu, con un fazzoletto bianco al collo: la sua mano è lorda come quella di Macbeth, orribilmente armata, come quella di Macbeth: tutto il suo braccio è intriso in un colore da ’89.

Già chinano le corna, ristando: egli non li ha guardati negli occhi: li accosta a braccio disteso. E prova l’acume del ferro sulla cervice, dove sa, tra vertebra e vertebra; alza, dopo incontrato il punto, il coltello e lo vibra fulmineo: nel modo, direbbe Leibniz, del «minor male possibile». La bestia si accascia pesantemente: coi quattro zoccoli all’aria, riversa, gli occhi morenti, agitata ancora da stratti e da sussulti paurosi, senza attenuazione possibile.

Qualcosa di sacro si spegne, l’essere si adegua alla immobilità. Una nera polla dalla cervice, la stanchezza suprema.

Il secondo lavorante introduce nella ferita una bacchetta pieghevole, quasi un giunco, e la sospinge per entro la colonna vertebrale una quarantina di centimetri a spegnere i moti del cuore: gli ultimi sussulti della meccanicità nervosa accompagnano nella bestia moribonda questo provvedimento dell’uomo, un tremito si propaga fino agli zoccoli, poi tutto il greve corpo è inerte. L’organismo è ridivenuto materia: il costoso elaborato delle epoche, disceso di germine in germine traverso i millenni, è annichilato da un attimo rosso.

Sperimenti fatti con la pistola o con la fulgurazione han dato inconvenienti gravi, mi dicono, spreco di tempo. L’animale dovette soffrire, talvolta, durante alcuni minuti: fuggì ferito, ferì gli uccisori. Il «minor male» è nel procedimento adottato.

Tre padiglioni da trentasei posti cadauno costituiscono il macellatoio dei bovi: in un quarto si attende ai cavalli: in un quinto ai porcelli: un sesto è l’ammazzatoio dei vitelli. Poco capretto, a Milano, salvo che a Pasqua.

Dove si lavora ai bovini, un capo sala e un vice-caposala. Tabellazioni accurate assegnano per ogni animale il posto, la matricola, il proprietario. Due squadre di undici accudiscono, in un’ora e mezzo, alla macellazione e alla preparazione di 18 buoi cadauna, dando, in capo a quel tempo, le 18 bestie finite, pronte pel trasporto o la cella. Mezz’ora, poi, di lavatura e di riordino: quattro turni al giorno; ove occorra.

Sui diciotto, inanimati e distesi, gli undici si dividono il cómpito con ordine e con una incredibile celerità: chiazzati nelle vesti, intrise le mani e le braccia di sangue, hanno alla cintola una scatola di zinco in forma d’una rigida guaina: è la sede collettiva di due o tre lame assortite; ed ancora poi l’«acciarino» dove le affilano, ch’è come una lima lunga e rotonda dal manico di legno, quasi uno stilo od un’arma di riserva.

L’opera totale si suddivide nelle specializzazioni. Il sangue viene chiamato giù da un taglio alla gola e ne gorgoglia orribilmente nero, dapprima, in bacili di zinco; vuotati questi ancora fumiganti in una cisterna di raccolta montata su carrello. Un altro operatore spicca la testa e le zampe, appende la testa al gancio zincato d’una specie d’attaccapanni: e quella ti guarda ora dai semichiusi occhi, immoti e vitrei come d’un cornuto Oloferne.

Di poi il corpo viene agganciato posteriormente, dalle ginocchia mozze e scoperte, i due ganci fra tendine e osso; ed è sollevato mediante un verricello, di cui le ruote superiori corrono sulla rotaia a mezz’aria. «I faccettisti» aprono l’animale ed estraggono i visceri; uno apre, uno estrae. Passano rapidi da un animale all’altro, affilando nel breve intervallo i coltelli. Quello che apre disegna prima il gesto col ferro sopra la pelle, quasi prendesse la mira, perché il taglio deve riuscir fermo ed esatto. L’eviscerazione d’ogni bove richiede poco più d’un minuto: aperto l’addome, ch’è in alto, la grossa polta delle trippe se ne riversa, e decade turgida, e talora verdastra, dilatandosi sul pavimento, gonfia di indesiderabile sterco. I trippai accorrono con speciali carrelli, piovuti come avvoltoi sulle ventraglie, e par che le rubino di tra i piedi agli operatori, asportandole verso i loro calderoni fetenti.

Seguono la scuoiatura, le operazioni di «abbellimento». La prima viene eseguita da cinque lavoranti sugli undici: uno «scalfa» i quarti di dietro, due imprendono invece a scorticare le due metà della pancia, fianco destro e fianco sinistro, e vengono detti doppioni. Due lavorano la schiena a distaccarne il «groppone», salito il primo sopra un alto sgabello: l’altro lavora dal basso.

La scorticatura è un’operazione delicata, intesa a cavar di dosso alla vittima la di lei pelle, senza sciuparla; la pelle è assai ricercata, venduta a un prezzo che ammonta fino al 10 per cento del valore totale della bestia. Così dèvesi evitare ogni «rigatura» o mala raschiatura che ne possa invilire il prezzo di vendita; incidere il connettivo soltanto, che la lega al grasso ed al mùscolo.

A Milano si opera la scuoiatura con coltelli ordinari a larga lama per le parti ondulate, con le scuoiatrici elettriche Bignami per le pance e le schiene. L’operaio si butta in ispalla uno speciale telaietto a zaino con il motorino elettrico, il giro-moto viene trasmesso alla scuoiatrice da un tubo snodato. La scuoiatrice ha forma d’un largo e piatto anello del diametro d’una dozzina di centimetri, provveduto di Manico. Delle lamette tipo rasoio girano celermente fra i due paralame anulari affacciati.

Una volta macellata la bestia, e scuoiatala, si procede al suo abbellimento.

L’«abbellimento» è una sagace preparazione dell’animale perché figuri netto e generoso di carne, senza pendule bacche di grascia o frastagliamenti di tèndini. Il coltello non è ormai che il pettine o l’arricciabaffi di un parrucchere ambizioso.

Certe drupe, certi strati di bel grasso compatto nella regione spaccata dello stracùlo vengono cincischiati vezzosamente a punta di coltello: un’acconciatura per il ballo di mezza quaresima.

Sopita l’angoscia, l’animo ormai si distende in una mattutina veduta di beccheria, nomi intesi ogni qualvolta in cucina rampollano dalle aperte costate, messaggeri del pranzo.

Il coltello agisce rapido e conscio: e va d’attorno leggero leggero ai ritocchi, un panno deterge dai carnicci e dal sangue la liscia parete del muscolo striato di chiari tendini; e poi l’ascia imprende a lavorare sommessamente, del macellarone più alto, che pare insonnolito sul mestiere: (ma che sa dove dare del taglio). Egli fende la colonna vertebrale con simmetria rigorosa, aprendo fra le due mezzène una finestra soltanto, che le lasci ancora congiunte, per buona figura, presso le culatte e la spalla.

Uno degli undici, con un grembiule anatomico e una sanguinosa borsa di cuoio semiaperta davanti, trascorre intanto da un animale appeso a quel dopo, lesto ladro fra le occupate menti degli altri: defrauda in un baleno le bestie delle loro ghiandolette essenziali, ipofisi, timo, surrenale, tiroide, paratiroide: e alle vacche gli ruba subito le ovaie.

Ogni testa cornuta, appesa al gancio con il «linguino» già fatto, egli la solleva di un poco mettendoci sotto la sua stessa testa imberrettata alla diàvola, pontando del corpo lavorando con le mani, col ferro e con gli occhi all’insù, come a staccare il batacchio d’una campana: e ne spicca invece qualche moruletta rossa, appiccicosa e molliccia.

In pochi minuti la sua borsa di cuoio è piena di opoterapia: i più autorevoli farmacologisti ne caveranno tiroidine e ovarine e preparati di ogni maniera, normalizzatori d’ogni più periclitante sistema endocrino. Pancreas e aggeggi del toro vengono acquistati a parte, dato anche il volume, per dedurne pancreatina e insulina regolatrice del tenore di zùcchero, o l’essenza quinta di una maschia generosità del pensiero.

Zitellone redente dall’acidità (di stòmaco) e diabetici ridivenuti amari come il calomelano devono a questi dieci minuti di lestezza e di previdenza la recuperata salute.

Ma ci vuole una formula! Sentito il parere del distillatore di formule, gli opoterapisti ne distilleranno mirifiche fiale; barbugliando in pentole senza precedenti storici le loro fantasiose decozioni. Le tre fatidiche sorelle compiranno il supremo incantesimo della vita, zoccolando d’attorno la caldaia a cavalcioni d’una scopa, in un ritmo ossitono da diavolesse:

Double, double toil and trouble:
Fire, burn: and, cauldron, bubble.

Tuoni e lampi! La più scarmigliata vaticinerà nello specchio discendenti maschi per otto generazioni ininterrotte a chiunque avrà comperato e pagato quel filtro.

E il lavoro continua, raddoppia. Le pelli, sùbito, ai commissionari delle concerie.

Tra un’ala e l’altra d’ogni padiglione è un andito ampio, coperto: vi vengono pulite, arrotolate, pesate, imbarcate. Il sangue, sùbito, che ancora fuma dai carrelli, a un attiguo e recentissimo impianto, che ne deduce concimi, lavori plastici, ornamenti, collanti.

È venduto fino a 120 lire il quintale. Intanto un andirivieni di garzoni: e alcuni omacci con la catena d’oro sulla pancia, che hanno l’aria di sapere perché son lì. Uno del Municipio collauda di timbri violàcei le bestie, ancora appese dopo ultimata la toilette. Mentre le doppie mezzène vengono carrucolate al frigorifero per il deposito e la frollitura, gli autocarri dei macellai si colmano d’altre mezzène e di quarti attingendoli dal frigorifero stesso o direttamente dai padiglioni: ingombrando tutta la lunga galleria di caricamento che divide quello da questi, dove incurvi garzoni trasferiscono a spalla tutto il meglio che possono, profumati quarti e mezzène, spalancati vitelli.

Li avevo persi di vista, creature della tepida innocenza, al triste limite dell’ammazzatoio, davanti i cucinoni maleolenti delle trippe, la loro anima pàrgola già quasi vanita nell’obbedire, prima ancora che l’uomo alto li mazzerasse alla nuca, senza lamento.

Neppur cadono, quasi: paiono ruzzare ad aggomitolarsi in un gioco. Vengono agganciati agli zoccoli dietro, sollevati meccanicamente sopra una vasca, sgozzati: il bruno orrore sgorga oramai da un oggetto.

Tutta la bisogna non richiede cinquanta secondi: preciso e infallibile è l’operaio dalla mazza, preciso e certo quell’altro che deve servirsi della lama abominevole.

Nuovamente carrucolati lungo le rotaie pensili fino ai singoli posti di lavorazione, dapprima un operaio li incide rapidissimo all’umbilico: ed applica poi nella ferita l’ugello d’una manichetta ad aria compressa, insufflandovi quanto ci vuole per gonfiarli a dovere, come dei maiali. «Parevano tanti cani appiccati», ed ecco in un attimo sono già gonfi: turgidi e netti: la lavorazione riesce più precisa sulla pelle e sulle carni distese, la punta e la lama incideranno più pronte i tessuti.

Ed ecco i compressori del frigorifero, che imperturbati motori vengono azionando nella Centrale pulita: coperti di candida neve sulle tubazioni d’espansione. Il frigorifero comprende un deposito generale del Consorzio per sosta fino alle 24 ore (già computata nel forfait di macellazione) nonché le celle dei singoli signori macellari. Temperatura ideale del deposito: cinque, sei gradi sopra lo zero.

Bianchi veterinari si aggirano per i padiglioni alla visita ultima, esaminando visceri e carni: chiedono a prestito un ferro, incidono, scrutano. Frequente la tubercolosi, massime per i bovini di stalla: e si rivela con caratteristici nòduli alla superficie dei polmoni e all’interno delle due pleure, talvolta è manifesta nel rene, nei vasi linfatici.

Allora i visceri vengono inviati alla sardinia, bolliti in autoclave seduta stante, degradati a materia e concime nella verde quiescenza della pianura. I veterinari si trasferiscono in bicicletta da un padiglione all’altro, vegliano a che nulla di sospetto abbia a varcare le chiuse barriere del macello: investiti del fidecommisso di una città e d’un popolo, la loro opera si esplica in un’attenzione continua, che vieti il male: constatandolo e distruggendolo davanti le porte della città.

Ruit hora. Il mercato del bestiame vivo e delle carni, nel suo clamore pieno di omaccioni, raccoglie alle strette di mano e ai buoni patti la folla dal mestiere impellente: negozianti, macellai, commissionari (le tre categorie tipiche): più qualche mediatore superstite ai tempi, che agisce per conto di una macellaia femmina padrona di negozio. (3)

Taluno della provincia ha un fazzoletto al collo, il cappello all’indietro. Sùdano, bofónchiano, annotano adagio adagio i suoi pesi e i suoi costi in un calepino bisunto, che fa le orecchie, con un lapis nero senza punta che a me farebbe venire subito il nervoso: e per loro, invece, è proprio quel che ci vuole, amico intimo dei mozziconi di «toscano» in fondo a una tasca.

Già gli autocarri strombazzano, chiedono il passo ai più grevi, nella galleria di caricamento dove ognuno tira a cavarsela quanto più presto gli è dato: fremono già d’irrorare del suo giusto vitto (per la dimane) la città che precipita oggi al suo giusto mangiare, verso i dodici tocchi.

Alcuni pochi sono dei baldracconi sfiancati, sanguinolenti, col tetto a pioventi, d’un color verde municipale 1888: altri scivolano via lisci e laccati di bianco, modernissimi, ermetici: fuggitivi ai lontani spacci e negozi.

Undici tocchi: e tutta una filologia scaturirà nel negozio tra la bilancia e la cassa, tra il garzone di banco e la serva, tra l’accetta e il libretto: (4) una nomenclatura conclusiva e perentoria, combinata di punta, di lonza, di canetta, di aletta, di scamone, di bamborino, di fiocco, di magatello, di filetto, di fesa, di culatta, di polpa. Ogni storia si adempie e si determina in una filologia.

La complessa organizzazione del Macello Pubblico comprende un importante reparto di microscopia che fa capo all’Ufficio municipale d’Igiene e di Sorveglianza Veterinaria: e infine una scuola per allievi macellai. L’esame clinico degli animali vivi, delle carni e dei visceri viene così ad essere fiancheggiato da indagini microbiologiche sulle carni stesse, sui sieri, sul sangue. Si isolano e si perseguono mediante cultura bacillare ed analisi microscopica i germi de’ mali infettivi, del carbonchio, ad esempio, della morva, dell’afta. La presenza della trichina, il microscopico verme che infetta le carni del maiale e dell’orso, è confermata mediante proiezione luminosa traverso la lastrina del preparato. Un apparecchio fotoscopico palesa con chiara evidenza sul telone a muro i gomitoletti insidiosi de’ vermi, che appaiono come rannicchiati tra fibra e fibra, quasi fossero a pensione dentro il fascio muscolare.

La scuola dei garzoni macellai, sorta nel clima del buon volere fattivo ad opera del Consorzio e del Municipio, auspici il Sindacato di categoria e la Federazione Provinciale del commercio, è intesa a munire di un qualche fanale conoscitivo i velocipedastri dal camiciotto rigato e dal collo rubizzo che sogliono pioverci addosso nelle vie di città quando meno ce lo aspettiamo, alati messaggeri di ossobuco, lacett e rognon. Comprende due corsi, primo e secondo, dove nozioni pratiche sul bestiame da macello, sui suoi pregi e difetti, sui modi di constatazione di essi, sulle razze tipiche, sull’allevamento, sulla tecnica del mercato, sul taglio, sul computo delle rese, sull’utilizzazione de’ ricavati, sulle malattie più frequenti, sulle qualità delle carni, sui nomi d’uso nelle diverse piazze, ecc. ecc., vengono impartite dal signor Gaetano Bestetti con chiara voce e intelligente tranquillità d’animo. Egli ha tra mano il bastoncello de’ «cacitt» e se ne serve come il geografo della bacchetta a individuare sui corpi appesi i singoli organi, i tessuti, le parti. Esercitazioni di taglio (come per i tagliatori sarti) completano il corso, su qualche maiale o vitello o quarto di bue, agganciato ad uno speciale cavalletto verde pieno di opportunità didattiche.

I due libri di testo, di prima e seconda classe, sono molto chiari, conclusivi, e ben fatti.

Alla fine del corso alunni e docenti si raccolgono in gruppo per una gioviale fotografia collettiva sotto il sole di giugno: e la Cassa di Risparmio delle Provincie Lombarde conferisce lire 150 cadauno ai diplomi di primo grado, 100 ai secondi, 50 ai terzi.



1. Ai limiti della campagna, nella zona periferica esterna dove ebbero sistemazione i macelli, sorgono case recenti, a sei piani: già cittadine e purtuttavia isolate: assai brutte, nei fianchi scialbati e nel tetto, in paragone delle vecchie cascine lombarde che i filari de’ pioppi e dei salci quasi nascondono, non fosse il fumo d’un camino a tradirle. Queste cascine, regolarmente distanziate l’una dall’altra, segnano la vecchia misura e la necessaria «giurisdizione» agricola della pianura lavorata.

2. Intendi: l’energia elettrica viene trasformata alla tensione di distribuzione; ch’è assai minore di quella di trasporto.

3. I negozianti vendono o commerciano bestiame in proprio: i commissionarî trattano per conto di terzi, cioè ditte importatrici o allevatori lontani: i macellai sono acquirenti, con bottega in città, e rivendono al pubblico.

4. Secondo il vecchio costume dei milanesi, il macellaio vende a credito, alle famiglie agiate: l’acquisto giornaliero viene segnato (marcàa) in un quadernuccio rilegato d’una teletta di poco prezzo, nera o rossa o azzurrina; sul fronte, impressa in oro, una testa di bue cornutissimo. Il regolamento del conto si fa a ogni fine mese. Il quadernuccio si chiama el librett, ed è uno dei pochi libri che ornino di lor presenza le case degli agiati lombardi. 