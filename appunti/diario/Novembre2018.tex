\section{Novembre 2018}
La verità è che non riesco a scrivere, vorrei tanto farlo, ma non ci riesco. Per me è difficile ora esprimere quello che mi passa per la testa, ho un po' di confusione e di preoccupazione.
\newline

\subsection{Gadda fascista}
Oggi ho letto \href{http://www.gadda.ed.ac.uk/Pages/resources/walks/pge/fascismdonnaru.php}{(Fascismo - Raffaele Donnarumma)}.
Leggendo queste righe mi stupisco del fatto che Gadda potesse essere stato fascista o per lo meno simpatizzante. Probabilmente essendo un reduce, vivendo quelle che all'epoca sembravano delle ingiustizie per l'Italia, aderì all'idea. Dopo come è scritto capì realmente cos'era il fascismo, precisamente in occasione della guerra di Etiopia.\newline
Oggi rivedo molti in buona fede che cadono in questa trappola, pensando che queste nuove forze ``sovraniste'' siano la cura a tutti i mali del nostro paese.
Invece questi politici cercano solo di ottenere voti dicendo alla gente quello che vuole sentirsi dire, trovando dei nemici e raccogliendo il consenso facendosi vedere in luoghi colpiti da maltempo e altre catastrofi per far vedere che loro lavorano e che si preoccupano, proprio come faceva Mussolini quando andava a mietere il grano.
\newline
Per me è avvilente vedere tutto ciò, vedere persono che accusano gli immigrati di essere un cancro, razzismo e aggressioni.
Il 3 settembre 1923, Gadda scrive da Buenos Aires alla sorella:

Tutte le settimane c’è una riunione al locale fascio, che attraverso mille difficoltà cerchiamo di consolidare, e io, come membro del Direttorio, sono impegnato a presenziarvi: dalle 9 alla 1 di notte. Le difficoltà, intendiamoci bene, non sono quelle di carattere «eroico» dei fasci in Italia, ma hanno invece la tinta intrigante e pettegola adatta alla microcefalia della colonia. | I giornali italiani di qui, primo fra tutti la Patria sono i primi denigratori del fascismo e per questo io non vi ho più scritto e non vi scriverò più fino a che la Patria non cambi bandiera. (Gadda 1987b: 85-86)

Poi, commentando «le notizie dell’assassinio della delegazione italiana ai confini albanesi», aggiunge:

Speriamo che il senso di responsabilità e di misura di Mussolini, la sua rapidità d’azione e la sua energia, facciano trionfare, come merita la ragione d’Italia.

Il lettore del Pasticciaccio, che ha in mente i vituperi contro il «Mascellone ebefrenico», il «Testa di Morto in bombetta», il «Merda», ha di che rimanere stupito; e questo stupore merita qualche risposta.

I motivi della convinta e precoce adesione di Gadda al fascismo (Hainsworth 1997) stanno nel nazionalismo, nell’interventismo e nel dannunzianesimo giovanili; confermati, dopo la guerra, dal risentimento del reduce e del borghese declassato, insofferente di «asinerie» e conflitti sociali (RR I 1197). Questa posizione, piuttosto tipica, è articolata e complicata dal Racconto italiano (1924-25). Il protagonista Grifonetto, a cui Gadda presta riconoscibili tratti autobiografici e i cui discorsi trapassano volentieri in quelli del narratore, si avvicina al fascismo per «idealismo» (SVP 398): il movimento di Mussolini è una «reazione netta, pratica, umana contro il nodo-gordiano della balordaggine ideologica accumulata dal secolo 18.o e 19.o» (SVP 417): balordaggine in cui sono conguagliati sia la «manìa fantastica delle palingenesi chimeriche» del socialismo, colpevole di aver idolatrato la «plebe» ignorandone le contraddizioni, sia il clericalismo, sia la grettezza dei «saggi borghesazzi» (SVP 484), sia il declino di un’aristocrazia che si è «rammollita» per darsi alle «opere buone» (SVP 566-7); insomma, tutta l’Italia giolittiana e liberale.

Il dannunzianesimo di Grifonetto come pure dell’ideologia di Gadda (inutile cadere nella trappola di distinzioni narratologiche) appare qui confermato; mentre è di matrice positivistica il legame fra la celebrazione del «lavoro italiano» e il fascismo, considerato non solo come forza nazionalistica, ma anche come motore di progresso, di ordine, di modernizzazione (elementi, per inciso, in cui si vuole anche vedere la milanesità di Gadda). Alla violenza fascista, il Racconto non oppone censure: essa è un portato della realtà storico-sociale e, come tale, inevitabile; spesso è una legittima risposta alle aggressioni di anarchici e socialisti (SVP 528-30, 563, 568); e semmai esprime drammaticamente l’assenza di spirito critico, l’«italianesimo (eccessività)», ma insomma anche il vitalismo generoso e puro che si incarnano in Grifonetto (SVP 484).

Resta una contraddizione, destinata a manifestarsi ben più tardi: la cialtroneria, l’esaltazione di una plebe che «è pura, è bella, è sana, è santa, è saggia, è intelligente, è sensibile, è eroica», un irrazionalismo che «non è scienza, non è filosofia, non è metodo» e che Gadda rimprovera nel ’24-’25 al socialismo (SVP 566), verranno a galla sempre più nel fascismo e diventeranno gli obbiettivi espliciti della furibonda satira di Eros e Priapo e del Pasticciaccio. Come spiegare allora questo rivolgimento? E quando collocarlo?

La prima parte del Castello di Udine ribadisce le radici conservatrici, militaresche e nazionaliste del fascismo di Gadda, sebbene in termini che evitano l’attualità politica. Eppure, fra il 1931 e addirittura il 1942 Gadda scrive una serie di articoli a celebrazione della guerra d’Etiopia, dell’autarchia, delle opere pubbliche promosse del regime, delle persone di Alessandro Mussolini e del generale De Bono (Dombroski 1974, 1984, 2002; Greco 1983). Non è solo fedeltà ai miti positivistici: in una paradossale estraniazione, Gadda ripete i topoi corrivi della propaganda del regime, e neppure rinuncia, in un sinistro pastiche, a imitarne la retorica.

L’attività giornalistica in cui nascono quegli articoli è la stessa che produce i pezzi degli Anni e delle Meraviglie d’Italia: libri il cui lindore letterario, e i cui silenzi, velano di lirismo prese di posizione altrimenti troppo dirette, e che di fatto sono tributari e beneficiari di un clima. Dal corpo delle Meraviglie, tuttavia, si stacca un racconto, presto dilato in romanzo: la Cognizione del dolore. L’interpretazione che legge in essa una sia pure cifrata satira del fascismo è fondata da Gadda stesso, che avverte nel ’63 di aver presagito «fin dal 1934-38» le «calamità» abbattutesi sull’Europa «dal 1939 al 1945» (RR I 759); e che, in un’intervista del ’68, spiegherà che «i vigili notturni» dei Nistitúos «sono visti come fascisti» (Gadda 1993b: 171). Si è dubitato, e ragionevolmente, che simili intenti polemici fossero davvero presenti nella stesura del romanzo (Manzotti 1996: 246). Ma il problema può essere posto in modo più radicale: la Cognizione il libro di un fascista? ha contatti con l’ideologia fascista?

Nonostante l’iscrizione al Fascio di Roma sino al ’39, l’attività pubblicistica e l’idillio letterario delle Meraviglie e degli Anni, la Cognizione rivela un dissidio che non può essere sanato. In Gonzalo lo spregio superomistico di Grifonetto decade a melanconia atrabiliare. Se la «tragedia di una persona forte che si perverte per l’insufficienza dell’ambiente sociale» (SVP 397) conduceva Grifonetto al crimine, qui il crimine è solo potenziale, e il personaggio è murato nell’abulia. L’odio che Gonzalo rovescia imparzialmente su borghesi e contadini ricade su quell’Italia che il velo allegorico del romanzo maschera.

E se il vero bersaglio è una società di massa che ha prostituito i valori, allora il fascismo è sotto accusa in quanto espressione di quella società e promotore di quella degradazione. Così, la ricerca di riparo e solitudine di Gonzalo è quella torre d’avorio costruita dai letterati fiorentini degli anni Trenta, ma senza ricomposizioni e anzi lacerata da una nevrosi storica (Luperini 1987/88). Questa oggettiva dissonanza con il regime (che, beninteso, è un sintomo di una generale dissonanza con il mondo) non può certo far concedere alla Cognizione una patente di antifascismo politicamente meditato; ma certo rivela il disagio di un reazionario che, dopo aver creduto nel fascismo ed essersi tappato il naso davanti alle sue storture, è travolto ora dalla sua follia.

è solo la catastrofe bellica a far maturare il distacco che, in un articolo del 1943, prende la voce di un trattenuto sarcasmo. Dal 1944, con i Miti del somaro, il Pasticciaccio ed Eros e Priapo il disgusto chiarisce le sue ragioni: Gadda vede nel fascismo una crisi di psicosi collettiva, e denuncia in Mussolini l’artefice del disastro. Ma di fatto, la storia si dissolve in un trattato di psicopatolgia freudiana (Amigoni 1995a: 78-98) che cancella ogni responsabilità politica. Resta un senso di colpa per la propria cecità; quello che induce Gadda a mentire spudoratamente, ancora nell’intervista del ’68 citata sopra:

Solo nel ’34 ho capito cos’era il fascismo e come mi ripugnasse. Prima non me n’ero mai occupato. Le camicie nere mi davano fastidio anche prima, ma era un fastidio e basta. D’altronde il libro Eros e Priapo l’ho scritto nel ’28 e mostra tutta la mia insofferenza per il regime. Ma solo nel ’34, con la guerra etiopica, ho capito veramente cos’era il fascismo. E ne ho avvertito tutto il pericolo. (Gadda 1993b: 168)

Scuola Normale Superiore, Pisa
\subsection{17 Novembre 2018}
Da anni prima a Milano e poi qui a Lecce ho frequentato palestre di sport da combattimento e poi di CrossFit. Ho cominciato ad allenarmi a 33 anni, molto tardi, ero molto ingrassato circa 115 kg, mangiavo quello che capitava e molta pasta. Un giorno mi venne il così detto fuoco di Sant'Antonio, mi ripromisi che dopo la guarigione avrei cominciato ad andare in palestra, così fu. Cominciai a praticare kick-boxing in una palestra di Milano, da li conobbi il CrossFit, l'allenamento funzionale. Dopo alcuni anni ho cambiato palestra, per poi approdare ad una accademia di jiu-jitsu brasiliano.
In realtà l'ambiente non mi piaceva, alcuni ragazzi erano arroganti altri invece simpatici. Vigeva una sorta di scala gerarchica fatta per anzianità, i più anziani e più forti nella lotta erano i più rispettati e in qualche modo esercitavano una certa autorità sugli altri. Alcuni cercavano di insegnare ai nuovi, di farli sentire più accettati e di fare amicizia; un sociologo definirebbe tutto ciò come un'\emph{istituzione totale}.\newline
Ma la cosa che più mi inquietava era l'atteggiamento del maestro, che alla fine della lezione dispensava consigli di vita e di comportamento. Diceva in sostanza come tutti avrebbero dovuto comportarsi e vivere, sempre secondo lui. Stando lontano da li mi rendo conto del brain washing a cui mi sono sottoposto stando li, stare li non rende  più forti, l'intolleranza non rende più forti, copre solo le paure che le persone hanno. L'unica cosa che rende più forti è la disciplina e la costanza, nel lavoro e nella vita di tutti i giorni.\newline
Non mi interessa essere sopra l'uomo medio, mi iteressa essere migliore con il passare del tempo, voglio migliorare nel lavoro e nella vita. Non voglio lamentarmi sempre.\newline
Il mio obiettivo non è quello di elevarmi sopra gli altri come vorrebbe insegnare questo maestro, ma quello di migliorare me stesso. Di lottare contro i miei impulsi, di lottare contro la parte peggiore di me.
E' difficile a volte superare i propri istinti, quello che ci permette di farlo è la disciplina.