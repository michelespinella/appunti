
\section{28 Ottobre 2015}
Un uomo alla fine deve per forza fare i conti con se stesso, ad un certo punto. Per me quel momento è arrivato. Credo che questo sia il periodo più brutto che abbia mai vissuto, le mie certezze si sono sgretolate una dopo l'altra dal giorno che mi sono trasferito a Lecce. Il mio mondo si è rivoltato, i poli si sono invertiti caldo è diventato freddo e il sole sorge ad ovest.\newline
La vita qui è diversa tutti vivono molto piano, mancano i servizi e
molte cose che avevo prima ora mi mancano

\section{29 Luglio 2016}
In questo ultimo periodo l'Europa ha assistito ad un'ondata di
violenza che si è scatenata con l'attentato di Nizza e quelli in
Germania. In passato altri attentati si sono verificati come quelli
più eclatanti a Parigi.
Tutti questi eventi sono stati scatenati da persone di religione
islamica e per lo più con origini nord africane o del vicino
oriente. Subito lo stato islamico del levante ha rivendicato gli
attentati ed ha fatto sue le ragioni dei terroristi, sono stati
mostrati dei video in cui alcuni di loro giuravano fedeltà altri si
addestravano.\newline
I giornalisti hanno liquidato questi fatti come se fossero opera di
squilibrati che in realtà non avevano effettivi legami con ISIS, il
tutto è stato semplificato e le notizie sono erano solo centrate sul
fare ascolti e visualizzazioni sui vari siti (vedi Repubblica.it,
corriere.it etc.), evidenziando aspetti morbosi, screditando i killer
dicendo che erano depressi, omosessuali, alcolizzati e magari
fedifraghi. Insomma una pletora di insulti a persono che nemmeno
conoscevano e di cui nessuno nulla sapeva.\newline
Vorrei ora però raccontare un episodio che mi porto dietro dalle
elementari, che però secondo me fa capire un aspetto importante del
problema con cui oggi ci stiamo confrontando.\newline
Un giorno qualsiasi di Ottobre del 1983 un bambino ha varcato la
soglia della mia classe, si è seduto vicino a me, ha estratto dalla
cartella un quaderno, ed ha cominciato a finire un disegno che
rappresentava un grosso tacchino. Ne fui impressionato perchè quel
tacchino era disegnato benissimo, era anche colorato di blu e
viola. Quel bambino aveva i capelli neri, occhi scuri e
carnagione leggermente olivastra. Se ne stava seduto al banco, quando
le maestre entrarono subito lo indicarono e dissero che sarebbe
rimasto per poco tempo, una settimana perchè figlio di giostrai che
stavano montando gli intrattenimenti per la sagra annuale a Noventa di
Piave. Lo dissero con un certo spregio, mostrando insofferenza come se
lui fosse un peso di cui liberarsi una sorta di problema in più che si
andava ad aggiungere ad altri ritenuti gravissimi ed
insormontabili. Lui non fece una piega, non alzò nemmeno lo sguardo
per guardare chi lo accusava in fondo di essere un problema, se stava
li vicino a me a colorare il suo bellissimo tacchino.\newline
Così facendo le maestre fecero percepire questo ragazzino come un
diverso, come un problema, come una zavorra; naturalmente per gli
altri bimbi in quella settimana divenne oggetto di scherno e derisione
perchè uno "zingaro" e quindi un ladro e quindi uno da schiacciare
come uno scarafaggio. Non so perchè quando sento nominare questi
"deviati" che compiono attentati mi viene in mente lui e mi chiedo per
arrivare a fare ciò quanto odio hanno dovuto subire ed ingoiare? Io
credo che quei gesti non siano spinti solo dal mandato di uccidere i
kuffar. E' mia ferma opinione che questi gesti siano anche mossi dalla
sete di rivalsa verso una società che li ha rifiutati e scherniti.

\section{24 Giugno 2017}
Una delle cose in questo ultimo anno mi ha turbato non poco è la guerra in Siria e Iraq. Fin da piccolo rimanevo impressionato a vedere le immagini del Libano e dell'Iraq, mostrate al telegiornale. Sentivo sempre un senso di profonda insicurezza, quella violenza mi suscitava sempre la stessa domanda: Perchè un massacro del genere, è causato solo dalla religione?
Oggi dopo 30 anni rivedo tutto quel sangue, mi chiedevo sempre quando la guerra in Libano sarebbe finita. Ora la Siria, quando finirà?

\section{28 Luglio 2017}
Rileggendo queste pagine rimango impressionato dal ricordo di quel bambino che disegnava il tacchino. Come posso ricordarmelo ancora? Eppure ancora oggi conservo quell'impressione di disprezzo degli altri bambini e delle maestre stesse. Per quel \emph{diverso}, la cui unica colpa era essere di etnia sinta. Per anni e anni il disprezzo, le battute e gli sguardi hanno colpito queste persone, i rom, poi gli albanesi, i meridionali ed infine gli immigrati clandestini di oggi. Il risultato si vede proprio oggi sui social network, che sono diventati la pubblica piazza su cui sfogare frustrazionie disagio, delusioni personali e lavorative sui più deboli, coperti dal pseudo-anonimato digitale.\newline
Come possiamo pensare di vivere in una società tollerante se da piccoli ci comportavamo così, l'unico che sollevò il vero problema del razzismo applicato fu Roger Casement, quando scrisse il rapporto sul Congo. Il diplomatico inglese descriveva la situazione in cui versava il Congo, come i belgi trattavano i lavoratori, pagandoli con delle barrette di metallo.\newline

Oggi dopo anni noi, un paese del G8, riserviamo lo stesso trattamento medioevale a queste persone, che scappano da guerra e torture. In molti mi dicono che queste cose sono sempre accadute e che fa parte della storia, non mi interessa. Io vivo il presente ed il futuro e mi rifiuto di vivere con persone che pensano e dicono cose del genere. Vorrei fare qualcosa ma cosa e come, non lo so.
\section{16 Agosto 2017}
Quella mattina l'aria era irrespirabile. Da est soffiava un vento umido e caldissimo, un
misto di salsedine e sabbia. La colazione in albergo era sempre la stessa, frutta, dolci, toast; a lato
c'erano anche delle pirofile contenenti dei cibi salati, per soddisfare le abitudini dei clienti americani o
tedeschi. Alina mangi una brioche e bevve del caffè, dal sapore simile a quello solubile.
Raccolta una copia di "Gulf News" dalla mensola della lobby, si avviò verso l'uscita con il solito passo
svelto, salutò il portiere in divisa ed usc .
Subito sentì quel leggero mancamento, dovuto al passaggio dai venticinque gradi all'interno dell'alber-
go, ai quarantasei esterni, come una mano che prende la gola e stringe nè troppo piano nè troppo forte.
Passava in mezzo alle persone, pensando a quello che i russi quella mattina le avrebbero chiesto, pensa-
va soprattutto alle lamentele che avrebbero di certo avanzato visto il ritardo nella consegna di alcuni
moduli del software. Una situazione che poteva diventare critica visto il cliente avrebbe potuto chiedere
il pagamento di una penale di diecimila euro per ogni giorno di ritardo, si ripromise di controllare atten-
tamento lo stato dei lavori, una volta ritornata a Milano. Camminava tra quei palazzi semi-vuoti, per lo
più alberghi di lusso, altissimi. Ad un tratto alzò gli occhi, si ritrovò di fronte una donna per un attimo
le sembrò di essere di fronte ad uno specchio, ma i capelli dell'immagine riflessa erano nero corvino.
Un brivido le corse lungo la schiena sentì le gambe cedere e poi il buio.
All'ospedale americano il termostato posto nel corridoio segnava i canonici veticinque gradi Celsius.
Alina aprì gli occhi e si trovò di fronte Maxim che nel suo inglese imparato all'Università di Perm, gli
chiese come si sentisse. Non riusciva a spiegare che cosa fosse accaduto, l'unica cosa che ricordava era
quella donna esattamente uguale a lei, con la sola differenza dovuta al colore dei capelli. Il primo
pensiero che le era venuto dopo che si era svegliata fu quello di aver avuto un'allucinazione dovuta
all'eccessivo caldo, oppure uno sbalzo improvviso di pressione. Tuttavia una parte di lei, le stava
dicendo che quello che aveva visto era reale, che quella donna era viva e vegeta ed era uguale a lei.
Ma come era possibile lei era nata e cresciuta a Milano, suo padre era di origine ligure mentre la madre
era di Padova, entrambi si trasferirono in Lombardia prima degli otto anni si conobbero tra i banchi del
Liceo Galileo Galilei di Milano. Entrambi avevano viaggiato solo in Europa, al massimo nel 1987 a
Cipro per una vacanza con le figlie Alina e Chiara.
Lei però di viaggi ne aveva fatti, da quando lavorava per la Microlite era stata in almeno 20 paesi del
mondo, dal Canada alla Tailandia. La Microlite era un'azienda di consulenza informatica con sede
legale nell'isola di Man, vantava trenta sedi nelle pricipali capitali mondiali un EBITDA che nessuna
azienda di software avrebbe mai potuto raggiungere, in molti avrebbero voluto accedere alle strategie
ed al codice sorgente che veniva prodotto.
In quel momento tutti quei viaggi le stavano passando davanti agli occhi come se stesse cercando di
rivedere tutti i luoghi, tutti gli alberghi e tutte le persone che in diciotto anni di carriera aveva visto,
con cui aveva parlato e con cui aveva lavorato. Ma non riusciva a vedere nulla, non riusciva trovare
nulla di strano o di fuori posto. Maxim la fissava, le disse che aveva parlato con il dottore e che l'in-
domani l'avrebbero dimessa. Il medico parlava di uno sbalzo pressorio dovuto al troppo caldo, una cosa
comune tra gli stranieri che venivano a lavorare negli Emirati.
\section{19-21 Agosto 2017}
L'azienda Microlite, è una delle aziende produttrici di software che fattura di più, anche più di Google e Facebook. Ma non è nota su Internet e non si fa molta pubblicità se non su qualche sito specializzato o qualche fiera dell'Oil\&Gas.\newline
Fondata nel 1985 da un ex dirigente del dipartimento IT di Exxon Mobil.\newline
\subsection{Giuseppe Pontiggia: dieci anni senza di Luigi Grazioli}
Dieci anni fa moriva Giuseppe Pontiggia, nel momento in cui, dopo il successo di Vite di uomini non illustri (1994) e soprattutto di Nati due volte (2000), la sua opera e la sua autorevolezza culturale e morale avevano ottenuto un vasto e meritato riconoscimento anche in campo internazionale. Oggi parlando con giovani scrittori e critici capita di scoprire che la maggior parte non ha letto una sua pagina, e che alcuni nemmeno l’hanno sentito nominare. E anche chi lo ha letto e conosciuto e stimato ne parla sempre meno, a parte le celebrazioni ufficiali, e talvolta ridimensionandone eccessivamente l’importanza senza che si capisca bene perché. Io gli ero amico e lo ammiravo. Non è solo per un atto di doverosa memoria che penso sia opportuno tornare a parlarne.\newline

Pontiggia, nato a Erba nel 1934, ha rivelato una precoce vocazione letteraria che si è poi affinata  alla scuola di Luciano Anceschi e del “Verri”, da cui è nata la neoavanguardia negli anni ’50-’60. Pur condividendo con essa gli assunti di fondo di una critica ideologica del linguaggio, da lui intesa soprattutto come incessante smascheramento di ogni suo uso retorico e mistificante, e di una spiccata attenzione all’aspetto costruttivo della cosa letteraria (e quindi ai meccanismi formali e ai risvolti metaletterari che lui però ha sempre trattato, nell’opera narrativa, in modo indiretto), e conservando negli anni l’amicizia con alcuni suoi rappresentanti (Antonio Porta, Alfredo Giuliani, Giorgio Manganelli), se ne è però allontanato abbastanza presto. Non lo convincevano gli estremismi formali, che sconfinavano spesso nell’illeggibilità, e la forte politicizzazione; ed è stato proprio nel momento della cosiddetta crisi delle ideologie e del conseguente rapporto tra politica e letteratura, cioè a partire dagli anni ’80, che la sua opera e la sua figura pubblica, di alto profilo morale, fortemente impegnato in alcuni settori civili ma non schierato e attestato anzi in un territorio che poteva apparire di neutra distanza, hanno acquisito un notevole rilievo. Per inciso, sono forse le stesse ragioni per cui è meno letto oggi: ragioni che però trascurano, colpevolmente, il rigore e la qualità della sua narrativa e di gran parte sua critica (in particolare Il giardino delle Esperidi, 1984).
Uno degli elementi caratterizzanti tutta la sua scrittura è stata, al contrario dei neoavanguardisti, la ricerca di una leggibilità di prima istanza che però contenesse, stratificata, la maggiore complessità possibile di riferimenti e implicazioni, anche teoriche, e quindi di letture.\newline

Attenzione verso il lettore e esigenza non solo di riconoscibilità, ma anche di riconoscimento, che, dopo l’esordio già compiuto di La morte in banca (1958 ma scritto a 18 anni, nel 1952), segnano il lungo travaglio che ha portato dal notevole quanto complesso e difficile L’arte della fuga (Adelphi, 1968) alla scrittura più abbordabile e diretta, anche se per niente semplice quanto a precisione e densità stilistica, e alla forma più accattivante, anche se costruita a partire dal sistematico spiazzamento delle strutture del giallo tradizionale, di Il giocatore invisibile (1979), con il quale ha ottenuto, oltre a una favorevolissima ricezione critica, il favore di un pubblico sempre più ampio. Nel frattempo si era affermato come un consulente editoriale sempre più ascoltato, come critico, e come curatore e traduttore di classici, sui quali rifletterà per tutta la vita, traendone molti spunti per la comprensione e l’analisi della società contemporanea, e per delineare, dal punto di vista esistenziale, il perimetro di una moralità disincantata, pessimista nel fondo ma partecipe e attiva (anche se la tentazione della fuga resta sempre in agguato: cosa che però lui trasfonderà solo nei romanzi: per esempio nella spia del Raggio d’ombra e nel protagonista assente della Grande sera).\newline

Il legame con le urgenze della vita, a cui non sono estranee anche alcune vicende biografiche, è uno degli elementi fondanti di tutta la sua opera e ne detta il ritorno delle tematiche (la ricerca quasi sempre delusa della verità, la paura, il tradimento in tutte le sue inflessioni, le ipocrisie e gli alibi…) ma anche i modi di affrontarle, che quasi mai procedono per via diretta. Ogni opera si basa su una ricerca, ma il suo centro è vuoto, e se i personaggi vi ruotano attorno per esserne quasi tutti risucchiati o annientati, il critico e il narratore invece ne disegnano i contorni e lo definiscono edificando ai suoi margini il mondo che esso ha disertato e insieme contribuito a far sorgere e procedendo per focalizzazioni oblique, topografie indirette e deviazioni attraverso le quali giungere a quell’essenziale che si sottrae a ogni indagine e riconoscimento diretto.\newline

Diventa quindi necessario servirsi, per raggiungerlo, di tutto l’armamentario che, senza recedere un millimetro dalla lucidità, sfrutta ogni risorsa dell’allusione, della reticenza, del passo a latere e della divagazione. Nell’opera narrativa tutto questo si traduce nel blocco dell’azione, nella descrizione e analisi spesso a valenza autonoma di personaggi e contesti, in dialoghi feroci in cui l’oggetto di partenza si rivela solo uno spunto per una nuova tappa dell’eterno conflitto tra i dialoganti, o nell’elisione di tutti i momenti nei quali la narrativa tradizionale concentra la maggiore densità di azione e agnizione.\newline

Nel suo lavoro critico invece la divagazione non prende la forma di lunghe riflessioni ma di brevi o brevissimi paragrafi che spesso partono o finiscono con una asserzione, morale o estetica, preferibilmente di segno aforistico. Pontiggia amava la scrittura che resta incisa, come sul marmo dei classici: e allora la frase si prosciuga, la sintassi si semplifica e le parole si fanno più nette, militare la topologia della punteggiatura. La quantità e la qualità delle pointes a cui dà luogo questa scrittura è ciò che più colpisce a prima vista di Pontiggia, come la varietà dei percorsi attraverso cui vi giunge, anche se alcuni sono privilegiati: l’inversione, il ribaltamento e il paradosso, e viceversa la letteralizzazione e l’etimologia.\newline

E’ la parte di coazione a ripetere che è toccata anche a lui, che tanto era sorvegliato: lo si vede quando, raramente, esagera nelle spiegazioni e nei commenti, quando non resiste all’impulso dello svelamento, da voyeur affascinato dalla stupidità e dalla facilità del pregiudizio; quando il meccanismo del paradosso sulla banalità prende il sopravvento e guida la narrazione e lo sguardo critico anziché esserne prodotto. (Se l’ossessione di indagare la banalità non garantisce di andarne esenti, il contrario resta la strada più sicura per cadervi, però.) Ma è anche l’effetto di una generosità nei confronti del lettore e un credito a fondo perso verso la sua intelligenza. E infine è ciò che lo differenzia dal narratore tradizionale nel quale ogni frase è necessaria e insieme deve essere dimenticata nella successiva verso cui è proiettata (e quando funziona lo è), e ciò che invece lo ricollega alla narrativa in cui le digressioni “non rallentano l’azione, semplicemente la sostituiscono”, come dice di Fielding, ma come è caratteristico anche di altri autori da lui amati, in particolare Manzoni.\newline

Nella narrativa, dunque, la divagazione disegna il centro vuoto della trama, il motore immobile e assente su cui si affacciano senza scorgerne il nucleo di verità le azioni (o piuttosto le reazioni) affannate dei personaggi, che l’autore segue non tanto con lo sguardo che la vulgata presume gelido dell’entomologo, quanto con una procedura di chirurgo dei sentimenti, che sa di dover passare per il dolore per essere efficace e talvolta con il fastidio dell’insegnante di fronte alla pervicacia dell’errore nei suoi studenti che pure ama.\newline



Nei testi saggistici questo vuoto radiante è quello che avvolge la struttura a brevi paragrafi, autonomi e compiuti: è ciò che del percorso logico è caduto, i passaggi impliciti, deducibili e quindi, per la scrittura, non essenziali, trascurabili. E, come insegnano i classici, ciò che è trascurabile va trascurato. L’argomentazione dimostrativa viene ridotta, quando non del tutto espunta, a favore dello sguardo concentrato sull’oggetto centrale, che si staglia isolato. La struttura del testo preso in esame e le sue peculiarità sono indicate per brevi cenni e non sembrano avere grande importanza di per sé, a meno che non siano funzionali al discorso, il cui argomento è introdotto da considerazioni che ne indicano chiaramente la fonte in qualche passaggio o elemento strutturale o tematico del testo, ma che resta personale e emozionale. Più che oggetto di esame, in questo caso, sarebbe meglio dire che il testo è l’occasione del saggio, lo stimolo all’approfondimento di una riflessione che non lo dimentica ma nemmeno lo feticizza: il passaggio frequente all’aneddoto biografico o storico lo dimostra.

Se questo conferisce ai saggi di Pontiggia il loro tono inconfondibile, preclude però le sorprese di una lettura dettagliata e interna (con alcune splendide eccezioni, a dimostrazione di quanto fosse attrezzato e sottile Pontiggia quando vi si dedicava). Quando si avvicina al dettaglio del testo, questo è in genere di tipo linguistico, con preferenza per l’etimologia per mettere in campo la pluralità dei significati o giocare la radice contro l’abuso, ovvero perché suscettibile di generalizzazione o in quanto tic stigmatizzabile. L’etimologia è anche un’importante molla di sviluppo narrativo, e non solo nel caso macroscopico della lettera anonima che dà il via a Il giocatore invisibile che prende spunto proprio da un’errata etimologia, quanto soprattutto nell’attenzione alle parole, alle loro sfumature e ai molteplici usi e strategie nel gioco di inganni quotidiano che costituisce il fondale di quasi tutti i libri di Pontiggia, che molti personaggi e più ancora le diverse ma analoghe figure di narratori condividono, facendone oggetto di analisi, puntualizzazioni che a volte sembrano pedanti e invece si rivelano sempre decisive per capire una figura o un evento, e soprattutto dei dialoghi, sottili proprio perché quasi sempre maligni.

Per quanto decanti talvolta l’avventura della lettura, questa resta circoscritta all’atto privato, mentre è raro che la traduca in motore della sua scrittura saggistica, che resta invece prevalentemente esterna e focalizzata sulle priorità di chi scrive. E’ lo scatto della fantasia o della riflessione che conta, l’accostamento fulmineo, l’istituzione di analogie, il potere esemplificatore e l’applicabilità al contesto socioculturale o alla convenzione morale. Questo finirebbe per dire molto su di lui e meno sul testo esaminato, se Pontiggia non ci sorprendesse con osservazioni in apparenza minori, ben mimetizzate nel discorso o ad esso funzionali che, pur non essendo sviluppate, suggeriscono invece quanto attento e rispettoso sia stato il suo approccio. Inutile dilungarsi; e se proprio, ci pensi l’interessato: gli elementi sono lì.

D’altra parte per Pontiggia la letteratura, anziché trovare in se stessa la propria legittimazione, è importante perché, nell’osservanza delle leggi che essa si dà e nell’attenzione rigorosa al linguaggio su cui si fonda, tende fuori di sé e serve a vivere, con tutto ciò che di vago ma anche di intenso e pervasivo questo verbo implica. In modo analogo il verbo servire, lungi dall’essere una diminuzione del valore della letteratura, ne è un’esaltazione: come lo è essere al servizio di qualcosa che, essendoti superiore, non solo dà un senso al tuo operato ma ti fa essere al meglio delle tue possibilità, quali che siano. Grazie al cielo Pontiggia non era di quelli che sminuiscono difensivamente ciò a cui dedicano l’esistenza. La letteratura sarà anche un gioco, ma che dà poco solo a chi poco si aspetta da essa, per fare una variazione su una delle sue espressioni preferite (per esempio sul matrimonio) che peraltro tradiscono a sufficienza quanto poco, talvolta, egli si aspettasse dagli altri. Per chiunque lo giochi è più opportuno che sia esigente, sia pure con leggerezza.



Di qui l’essenzialità della componente saggistica e, sornionamente, sapienziale anche del suo lavoro narrativo: efficace però solo in rapporto all’economia linguistica, alla tensione espressiva che lo percorre sempre, anche quando la catena delle definizioni e degli aforismi si prolunga di un anello non proprio necessario per amore di completezza, o di un’ulteriore specificazione, per quanto acuta.
Di questo dobbiamo essergli grati; di certo gli sono molto grato io, per ciò che da lui ho imparato e continuo a imparare, senza mai annoiarmi, ogni volta che lo rileggo.

Narrare però, in qualsiasi accezione si voglia intendere il verbo, è un’altra cosa. Pontiggia ha ribadito in più occasioni che scrivere opere narrative per lui aveva valore solo nella misura in cui non sapeva dove stava andando e in cui scopriva ciò che sarebbe accaduto in corso d’opera, venendone sorpreso; eppure raramente in lui c’è abbandono. Il suo controllo anche su ciò che andava scoprendo era sempre lucido, senza cedimenti, direi quasi tirannico. E’ il suo stigma, lo sappiamo; è la sua forza, ma anche il punto in cui si irrigidisce: non cede mai. Tanto che vien da pensare a quanto circoscritto fosse l’orizzonte entro il quale la sorpresa potesse delinearsi. Quando non si cede mai, inoltre, il rischio è di schiantare tutto in una volta. Viceversa, chi cede è già schiantato una volta per tutte, avrebbe certamente ribattuto. E’ probabile, ma forse tra le due alternative (le stesse entro cui si muovono alcune delle pagine più acute di Pontiggia, che non a caso faceva del ribaltamento uno dei suoi strumenti retorici privilegiati) esistono altre possibilità. Può darsi che la mia lettura sia parziale, reattiva come lo erano alcune delle sue pagine critiche: appunto, ma a me sembra che queste possibilità Pontiggia le abbia più trascurate che coltivate. Può darsi che stia facendo l’errore di attribuire a Pontiggia un desiderio che è solo mio, che cioè lo stia accusando di essere quello che è e non quello che vorrei io (è una mossa che ovviamente Pontiggia non ha mancato di denunciare, come fanno gli scacchisti che pensano in anticipo le contromosse possibili degli avversari prima di muovere, col rischio però di suggerirle mentre si fa mostra di averle pensate); tuttavia a volte mi viene il dubbio che, prigioniero della logica dello svelamento e della difesa sistematica preventiva, non avesse l’immaginazione teorica e narrativa necessaria per pensarle, quelle alternative. Il suo è un mondo perfetto, catafratto, e insieme asfissiante per eccesso di difese. Mentre invece è appunto questo il difetto di immaginazione di Pontiggia: che in lui ben poco non sia voluto, anche la sorpresa, che a posteriori viene ricondotta nello stesso alveo. Parla della paura in molti testi perché è il primo a sentirla, ma mentre la affronta non resiste all’impulso di circoscriverla, anche nel senso di scriverle attorno. E’ così presente nelle situazioni e nei personaggi, ma meno nella scrittura. C’è un eccesso di intelligenza (che certo è sempre meglio del contrario), ma deriva come da un eccesso di timore della stupidità: nel senso, qui, di ciò che sfugge al controllo e aggira anche l’attenzione più minuziosa, la strategia più paranoica.

Già, la stupidità, quella che per esempio nei cattivi romanzi si traduce nell’adozione di formule narrative e stilistiche già usate, come se fosse possibile evitarle del tutto (lo si vede già da L’arte della fuga: un titolo che suona come un programma per tutta la vita, la definizione precisa di una strategia che solo tardi e a fatica Pontiggia cercherà di abbandonare; per ogni evenienza spesso dirà che il nuovo a tutti i costi è il falso miraggio delle avanguardie: salvo sentirne sempre l’esigenza come retropensiero respinto, come scatto ironico ogni volta che il déjà vu rischia di spuntare nei paraggi, - e per uno che ha letto moltissimo spunta nei paraggi quasi sempre). Credo che per Pontiggia fosse una vera e propria ossessione (comprensibile e sempre attuale d’altronde: basta guardarsi attorno; e anche dentro): non a caso ci ritorna in tutte le sue opere e ne illustra magistralmente tutte le sfumature che la comunicazione odierna, totalizzante, non si stanca di moltiplicare, anche se questo non sempre ha giovato alla sua narrativa, perché questa attenzione rischia di frenare, di rendere reattivi alle banalità sempre in agguato piuttosto che attivi, cioè attraversandole anche a rischio di restarvi impantanati.

La tipologia dei suoi personaggi è ciò che gli permette di muoversi in questa dimensione e di scandagliarla con la massima efficacia (e crudeltà), ma al contempo gli impedisce di uscirne. L’eccezione, oltre all’estrema conquista di Nati due volte (una conquista in primo luogo per lui stesso), è Vite di uomini non illustri, la cui nascita non a caso è stata una felice sorpresa anche per Pontiggia, un’urgenza improvvisa che ha dato libertà alla sua scrittura.

I personaggi appartengono infatti al ceto colto, spesso sono intellettuali o che hanno la presunzione di esserlo: l’estensione della loro stupidità è pari almeno a quella delle loro competenze, quella delle loro debolezze all’acume del loro sguardo su quelle degli altri: lo spazio in cui muoversi è pertanto molto ampio, come alto può essere il tenore di ciò che viene detto e taciuto, rivelato e misconosciuto.
La valorizzazione della propria intelligenza e delle proprie competenze è pari, in essi, alla misura delle lacune e delle debolezze dei loro antagonisti (perché tutti lo sono rispetto agli altri; raramente c’è complicità e armonia, e quando sembrano esserci, ci pensa il narratore a mostrarne l’illusorietà, quando non la deliberata menzogna), il che non impedisce a tutti di compiere sistematicamente delle scelte sbagliate, di stupidità sesquipedale, che qualsiasi persona di banale buon senso eviterebbe senza nemmeno pensarci.\newline

Bisogna dire che in gran parte se lo meritano, ma non meritano anche altro? Certo dal punto di vista narrativo qui comincerebbero i problemi, però viene il dubbio che è proprio dove ci sono i problemi che un narratore dovrebbe talvolta inoltrarsi. D’altra parte non è difficile per un narratore illustrare infantilismi e stupidità se si fanno compiere sistematicamente ai protagonisti atti stupidi o infantili, come nel caso del professore del Giocatore invisibile e del medico del Raggio d’ombra per esempio.
A volte mi viene da accostarlo al peggior Nabokov, che si diverte un mondo a demolire gli idioti a cui lui stesso dà forma e vita. Una debolezza ipercompensata da ben altre grandezze, sia chiaro. E però indicativa delle debolezze, anche umane (nel senso di disposizione interna alla scrittura, non come riferimento personale, che verterebbe solo sulle miserie del pettegolezzo, nemmeno sui fasti talvolta rintracciabili nell’aneddoto, mentre nell’analisi dell’opera sarebbe solo fuorviante), che pure in quelle sbucano qua e là.

Alla dinamica dello svelamento, della denuncia e del ribaltamento non c’è limite. Senza che questo lo renda inutile, ogni smascheramento della menzogna, della banalità e della stupidità, è suscettibile di diventare oggetto della medesima procedura. Il sollievo che ne deriva è l’inizio di un nuovo perturbamento, l’indizio di una fragilità strutturale che non si rassegna a se stessa e si intestardisce fino ad autodemolirsi perché non ha la forza di accettarsi. E’ un bene e un male. Con la chiusura della sentenza, il moralista decide quando è il momento di arrestarsi, ma non è detto che la sua decisione venga sempre condivisa. “Te lo dico io come stanno le cose”, sottintende, anche quando sospetta, o sa, che le cose stanno così solo nella misura in cui una decisione taciuta così le fa stare, cioè consistere. Ma le cose non stanno né così né cosà: è quella parte di me che, in quel momento e per qualche ragione, decide che le cose così devono, o dovrebbero stare, che le fissa per quel momento come se fosse per molti altri momenti, e magari per sempre. E’ la necessità dell’etica a esigerlo, sono la sua urgenza e il suo limite a imporlo. La falsità denunciata, o smascherata, della non decisione (che è l’attività prevalente dei personaggi di Pontiggia), si contrappone all’enigmaticità delle scelte, per quanto queste siano spesso sordide, come l’invidia, la gelosia o debiti di gioco accennati di passaggio, o reattive al gesto di altri o alla sua assenza, nate da una menzogna analoga e da una analoga non decisione antecedente; ma nessuna di queste scelte è vissuta fino in fondo, perché nessuna deriva da una vera necessità, tranne forse quella di Paolo in Nati due volte. C’è sempre sospetto, e poca compassione, cioè comprensione o simpatia per la debolezza (ancora con l’eccezione di Vite di uomini non illustri, che non a caso è un’eccezione nell’opera di Pontiggia).

E’ vero che compassione e simpatia, essendo passati da pietre angolari del romanzo originario settecentesco alla più abusata cassetta degli attrezzi del romanzo di consumo otto-novecentesco, sono in testa all’elenco dei sospetti in quello contemporaneo, specie dopo le avanguardie (e la fobia del sentimentalismo, doverosamente distinta dal sentimento, con relativa elevazione a potenza del sistema pudore-reticenza-ironia-sarcasmo, è un tratto che Pontiggia condivide con il novecento postavanguardistico, dal quale si è in parte separato dopo l’adesione giovanile), ma a volte viene da chiedersi se questo non sia un difetto dalla prospettiva di chi narra (al di là delle confessioni di prammatica circa la meraviglia di ogni vita ecc.), nel momento stesso in cui rifiuta o aggira, invece di affrontarla, una possibile pietra d’inciampo. Il fatto è che, ridotte all’indicibile, invadono poi il privato in una misura vergognosa: e infatti se ne vergognano tutti quanto più vi indulgono, e viceversa si disprezzano quando se ne scoprono del tutto incapaci, attestando, se non la miseria della loro vita privata, di certo quella del modo in cui la percepiscono.

Comunque sia ognuno cerca di scrivere di ciò che più gli importa, e finisce per scrivere solo ciò che può su ciò che sa, per lunga consuetudine esistenziale e intellettuale: per Pontiggia sono le finzioni sociali e famigliari, l’ipocrisia, il tradimento, il sospetto, le ambizioni e le velleità, l’infingardaggine, l’inazione. Ognuno ne tragga le deduzioni che vuole: il problema è che il sistema di opposizioni e divaricazioni su cui si basano resta impensato, mentre è chiaro l’orizzonte entro il quale agiscono nelle opere di Pontiggia, che è quello della scomparsa del “reciproco rispecchiamento tra nome e mondo”, e in fondo del rimpianto della loro coincidenza, se non addirittura del sentimento che la pensava possibile (che è un’altra forma dell’aborrito sentimentalismo). La possibilità di vivere la sua impossibilità non si dice euforicamente, ma nemmeno solo nella disperazione o nel rimpianto, non è prevista da Pontiggia.

Ancora una volta questa è la sua forza e la sua debolezza, cioè l’incapacità di pensare altrimenti. Il rifiuto di ogni placebo, cioè dei placebo noti, non sarebbe a sua volta che un’altra forma di placebo. E d’altra parte chi non ne ha bisogno? (Ma allora perché disprezzare chi si accontenta dei più diffusi e a portata di mano? Perché quando si insiste solo sulla loro menzogna e ipocrisia, il disprezzo non è distante.)\newline
Finché si resta nel limes da lui tracciato, il suo rigore, la sua intelligenza e sottigliezza abbagliano; se appena si spinge lo sguardo al di là, o sulla linea di confine, spunta l’insoddisfazione. A me è capitato talvolta a libro chiuso. Più raramente a libro aperto però: e questo è moltissimo.\newline
Non facendosi troppe illusioni sull’avvenire e sul divenire, ma essendo troppo accorto per farsene sul passato, Pontiggia indicava come alternativa la via che non è stata percorsa, forse perché non percorribile, almeno da noi: quella dell’essere.
Allo stesso modo, pur diffidando degli ottimisti, che sono cretini o ipocriti interessati (o entrambe le cose), era di converso troppo intelligente per scegliere apertamente il pessimismo, che ha quasi sempre ragione, ma solo a posteriori. Il pessimismo avrà anche la sua bella aura di saggezza, infatti, ma è una saggezza miserabile, troppo facile, in quanto trascura ciò che nel tempo viene raggiunto diventando subito acquisito, cioè irrilevante, mentre gli immancabili effetti negativi a lungo termine, più gravi quanto meno immaginati, vanno ad aggiungersi agli altrettanto immancabili errori immediati, e insieme bellamente imperversano. Così Pontiggia aveva optato per un bonario scetticismo, che non risparmiava uno sguardo caustico sulle debolezze di ciascuno ma conservava un fondo di benevolenza per lo stupore che ciascuno suscita, guardandosi bene però dall’esprimerlo se non come principio generale, perché sarebbe stata una debolezza eccessiva. E quando si espone il lato debole, all’interno di questa logica, immediatamente qualcuno si precipita a colpire. Quindi è meglio evitare di esporlo. Il rischio però è che, preoccupati di proteggere tutti i possibili lati deboli da ogni possibile attacco, come insegna la paranoia, non solo si perda a moltiplicare le difese il tempo che forse darebbe frutti migliori se impiegato altrimenti, ma soprattutto che diventi fragile l’insieme e cada tutto in un colpo solo, magari da sé per il suo stesso peso. Come rischia di succedere alla libreria di Perego, insieme alla torre che la ospita, nel Raggio d’ombra. L’invulnerabilità delle parti è la vulnerabilità dell’insieme. Totale e definitiva. Il pessimista mascherato risponderebbe: ma tanto quello è inevitabile, è il destino di ogni cosa!
Sì, ma intanto…
E poi: se si mette in conto, o addirittura si accetta già in partenza la sconfitta definitiva, perché non essere talvolta magnanimi, o generosi, con quelle minori? Pontiggia lo era certo di persona, ma nei suoi romanzi talvolta si fermava prima. Il passo decisivo in questo senso lo aveva compiuto con Vite di uomini non illustri e con Nati due volte. A proposito del primo aveva spesso parlato di una nuova serie. Purtroppo non ha potuto scriverla.