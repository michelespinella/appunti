\section{Note sull'Italia di oggi}
E' quindi giusto lasciare che tutti dicano quello che vogliono, pur sapendo che queste cose potrebbero risultare dannose per alcuni? Potrebbero cioè indurre masse di persone a credere a delle affermazioni che potrebbero danneggiare la loro salute o quella dei loro figli.\newline
Oppure è corretto che delle persone vadano in giro a dire che la terra è piatta?\newline
Comunque chi dice che questo tipo di pregiudizi e credenze è un problema di oggi, sbaglia. Le notizie false già esistevano durante la prima guerra mondiale, infatti ai soldati italiani sul fronte orientale veniva distribuita una foto che rappresentava un neonato, di provenienza austro-ungarica, con le sembianze di un diavoletto, con corna e coda. Ovviamente questa foto era un artefatto, tuttavia ha indotto moltissimi soldati italiani a credere che gli austriaci fossero dei diavoli. Oggi i più colpiti dalle notizie false sono gli immigrati, vengono infatti prodotte delle foto modificate con delle frasi che inducono a molti a credere a delle assurdità, un esempio lampante è quello dell'introduzione dei numeri arabi nelle scuole italiane. Tutto ciò è scatenato contro persone che scappano da guerre e carestie, che giunti in Italia vengono sfruttati come schiavi. E quando non riescono ad attraversare il canale di Sicilia o muoiono annegati oppure vengono torturati ed uccisi dalla polizia libica.
Ieri ho visto un video di Matteo Salvini (attuale ministro dell'interno) che per strada scherniva alcuni giovani immigrati seduti su un muretto. Non so a quando risalga questo video, ma tutto ciò è costruito ad arte per guadagnare consensi e per parlare alla parte più oscura delle persone. Infatti molti oramai hanno identificato gli immigrati come i colpevoli di tutto, basta quindi un post o un'affermazione fatta al momento giusto per scatenare la rabbia e l'indignazione. Una rabbia che ormai è alimentata dalla stampa che disegna ogni giorno scenari apocalittici trascurando magari notizie che potrebbero aiutare a comprendere meglio altri fatti.\newline
Ma quante volte queste cose sono state dette, quante volte si è sentito parlare di fake news, immigrazione e povertà. Troppe volte si parla del problema ma se ne ignorano le cause, i fatti che hanno portato a tutto ciò.
Roger Casement tra il 1911 ed il 1916, visitò l'allora Congo Belga in qualità di inviato dell'impero britannico constatò le condizioni degli abitanti dell'allora Congo leopoldino. Soprusi e massacri, lavoro pagato con barrette di ottone e razzismo costruito ad-hoc per danneggiare la popolazione autoctona. \newline
Vi sono altri esempi come il Congo, in cui gli stati europei si sono distinti per crudeltà e per le varie efferatezze fatte sulle popolazioni. Ora gli africani spinti dalle condizioni disastrose in cui versano, vengono in Europa. 
\subsection{24 Agosto 2018}
Oggi guardando il telegiornale, mi sono reso conto delle persone che vengono tenute, di fatto in uno stato di prigionia nella nave della guardia costiera "Diciotti". Ci sono per lo più Eritrei, Siriani e Sudanesi, già questo può far capire che stanno scappando da delle guerre. Non sono i palestrati con il cellulare di cui parla Salvini, andrebbero accolti e assistiti, le persone e i politici dovrebbero capire lo stato di fatto delle cose e non raccontare bugie gli uni e credere a tutto gli altri. Oramai la situazione delle notizie false nei social network è tragica, si raccolgono dei veri e propri drammi umani, persone che non sanno scrivere e non capiscono quello che leggono.\newline
E' degradante vedere tutto ciò, dopo la formazione di questo governo ci sono state delle aggressioni razziste in tutta Italia, tutti però dicono che si tratta di goliardate. Il primi a sostenere questa tesi sono i due vice primi ministri, secondo questi ultimi non esiste il problema, come quando Berlusconi negava l'esistenza della mafia. Per me è dura vedere consumarsi tutto questo, sto veramente male anche quando sento persone a me vicine che esprimono frasi discriminatorie, in primis mio fratello e sua moglie. Quello che penso è che le persone si stiano facendo distrarre dalla reale situazione economica, credono veramente che la colpa di tutto sia degli immigrati.\newline
\subsection{10 Febbraio 2019}
L'altro giorno guardavo un documentario su un r