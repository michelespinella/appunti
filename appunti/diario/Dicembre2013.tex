\section{22 Dicembre 2013}

Quanta ignoranza ed incompetenza, quanta arroganza di essere dalla parte del giusto. I dirigenti dell’azienda in cui sono impiegato sono completamente all’oscuro delle necessità dei lavoratori e a mio avviso anche dell’azienda.
In quell’ufficio regna il disordine, la sporcizia e il rumore. Serpeggia un continuo ed inarrestabile malumore, dovuto innanzitutto ai salari inadeguati rispetto le mansioni di ogni uno e ai carichi di lavoro.
Nei progetti la disorganizzazione oramai è cronica, le risorse non gestite. Potrei scrivere pagine e pagine di tutto ciò ma non è costruttivo,quello che comunque risulta da questa situazione è una sorta di guerra tra i colleghi, una faida intestina tra persone che lavorano allo stesso banco.
La competizione che spopola è quella di sapere lo stipendio di quante più persone. La spaccatura tra il gruppo dirigenziale e i lavoratori/tecnici è profonda, non esiste uno strato intermedio che guidi o gestisca. Tutto è lasciato al caso, non esiste una politica di gestione e di pianificazione.
Un altro comportamento che ho rilevato è quello di denigrare sistematicamente il lavoro altrui. Esiste infatti un gruppo di persone che detiene la verità “tecnica”e poi ci sono gli altri, i plebei, che sono ignoranti di cui farei parte anche io.
Questi atteggiamenti sono dovuti a superficialità, all’incapacità di rapportarsi con gli altri. Devo ammettere di non essere una persona facile, ma con il tempo mi sono reso conto che questo atteggiamento di contrasto non porta a nulla. Credo che la pazienza e la collaborazione siano le uniche “armi “ di cui una persona si debba dotare nel mondo del lavoro; far sentire le proprie ragioni urlando non porta a nulla.

%\JournalEntry{Domenica, 18 Ottobre 2018} - Got under way at 6 A.M.,
%and are now about half-way between Paris and Rouen.